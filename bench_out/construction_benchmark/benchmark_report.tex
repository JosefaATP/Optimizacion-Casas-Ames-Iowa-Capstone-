\documentclass[11pt]{article}
\usepackage[margin=1in]{geometry}
\usepackage{booktabs}
\usepackage{siunitx}
\sisetup{group-separator={,},group-minimum-digits=4}
\usepackage{graphicx}
\usepackage{pgfplots}
\usepackage{pgfplotstable}
\usepgfplotslibrary{groupplots}
\pgfplotsset{compat=1.18}
\title{Benchmark Construcci\'on: Comparaci\'on XGB vs. Regresi\'on}
\date{\today}
\author{Reporte autom\'atico}

\begin{document}
\maketitle

\pgfplotstableread[col sep=comma]{price_by_budget.csv}{\budgetdata}
\pgfplotstablecreatecol[create col/expr={\thisrow{avg_roi}*100}]{avg_roi_pct}{\budgetdata}
\pgfplotstableread[col sep=comma]{delta_by_neighborhood_top.csv}{\neighdata}
\pgfplotstableread[col sep=comma]{roi_by_neighborhood.csv}{\roitab}
\pgfplotstableread[col sep=comma]{roi_hist.csv}{\roihist}
\pgfplotstableread[col sep=comma]{price_by_lot.csv}{\lotdata}
\pgfplotstableread[col sep=comma]{features_vs_baseline_top.csv}{\featuredata}
\pgfplotstableread[col sep=comma]{mip_scenarios_summary.csv}{\mipscenarios}
\pgfplotstableread[col sep=comma]{mip_scenarios_roi_by_budget.csv}{\mipdetail}
\pgfplotstableread[col sep=comma]{sensitivity/scenarios.csv}{\scenarios}
\pgfplotstablecreatecol[create col/expr={\thisrow{roi_xgb}*100}]{roi_xgb_pct}{\scenarios}
\pgfplotstablecreatecol[create col/expr={\thisrow{roi_reg}*100}]{roi_reg_pct}{\scenarios}

\section{Resumen ejecutivo}
\begin{itemize}
  \item \textbf{Cobertura:} 28 barrios, 7 tama\~nos de lote (1\,300--31\,700~ft$^2$) y 4 presupuestos (400k--1M USD). De las 784 combinaciones posibles, 723 fueron factibles (\texttt{status=2}) y 61 quedaron infactibles, todas asociadas al presupuesto de 400k USD.
  \item \textbf{Brecha predictiva:} El XGB estima precios \emph{post-obra} en \num{507296}~USD promedio vs. \num{468394}~USD de la regresi\'on lineal; la brecha media es \num{38902}~USD (8~\%). En barrios premium como SawyerW y Veenker el gap supera los \num{80000}~USD (Figura~\ref{fig:gapNeigh}).
  \item \textbf{Rentabilidad:} La utilidad (precio--costo) media es \num{31693}~USD con ROI medio 8.6~\% (p25 4.9~\%, p75 11.8~\%). Un 12.6~\% de las corridas presenta ROI negativo (Figuras~\ref{fig:roiHist} y \ref{fig:negatives}), lo que permite filtrar escenarios antes de invertir en planos.
  \item \textbf{KPIs Informe 2:} Cada corrida guarda el porcentaje neto de mejora (\emph{Uplift}) y el porcentaje de presupuesto usado. Los agregados muestran que el tier 400k genera +3.2~\% de aumento frente a la regresi\'on consumiendo 91~\% del capital, mientras que 1M sube 9.5~\% pero usa solo 50~\% del presupuesto.
  \item \textbf{Slack presupuestario:} Aun en el presupuesto de 1M USD se conserva un slack medio de \num{504727}~USD, lo que abre espacio para imponer restricciones ESG, smart-home o amenities espec\'ificas sin sacrificar utilidad.
  \item \textbf{Perfil de producto:} El optimizador empuja las viviendas a calidades ``Above Average--Excellent'': m\'as \emph{Gr Liv Area}, s\'otano y garage que las medias hist\'oricas de cada barrio (Figura~\ref{fig:features}), manteniendo un n\'umero moderado de recintos.
\end{itemize}

\section{Dise\~no experimental y cobertura}
Todos los experimentos corrieron con los mismos hiperpar\'ametros del modelo MIP de construcci\'on, usando como referencia el pipeline de regresi\'on lineal y el modelo XGB entrenado sobre la base de Ames. La Tabla~\ref{tab:coverage} resume las combinaciones resueltas; las infactibilidades de 400k se listan en \texttt{missing\_combos.csv} para trazabilidad.

\begin{table}[h]
  \centering
  \caption{Cobertura del benchmark.}
  \label{tab:coverage}
  \begin{tabular}{lcccc}
    \toprule
    Barrios & Lotes & Presupuestos & Combinaciones te\'oricas & Corridas factibles \\
    \midrule
    28 & 7 & 4 & 784 & 723 factibles + 61 infactibles (400k) \\
    \bottomrule
  \end{tabular}
\end{table}

\section{Sensibilidad general de regresi\'on/XGB}
El pipeline de modelamiento se ejecuta en dos niveles: primero se valida que la regresi\'on y el XGB mantengan la misma jerarqu\'ia en los experimentos generales (remodel y construcci\'on) y luego se profundiza en el benchmark de construcci\'on. Esta secci\'on resume la capa transversal antes de entrar al detalle espec\'ifico.

\subsection{Contexto remodel vs. construcci\'on}
\begin{itemize}
  \item \textbf{Remodel:} Las corridas \texttt{remodel\_benchmark.csv} conservan los tiers low/mid/high (18k, 50k y 120k~USD) y sirven como \emph{stress test} del stack predictivo. El ROI extremo proviene de presupuestos peque\~nos frente a mejoras puntuales, por lo que se usan como filtro r\'apido antes de campa\~nas de remodelaci\'on.
  \item \textbf{Construcci\'on:} Las 723 soluciones factibles resumidas en \texttt{summary\_overview.json} muestran precios XGB promedio de \num{507296}~USD frente a \num{468394}~USD de la regresi\'on (brecha \num{38902}~USD, 8~\%). El ROI medio es 8.6~\% con slack medio \num{250538}~USD y una utilidad promedio de \num{31693}~USD, por lo que esta capa sirve como baseline para cualquier lectura posterior.
\end{itemize}

\subsection{Sensibilidad del dise\~no (predictores)}
El script \texttt{run\_sensitivity.py} genera barridos controlados sobre los predictores (archivos bajo \texttt{bench\_out/construction\_benchmark/sensitivity}). La barrida univariada \texttt{univariate.csv} muestra que, para el perfil base (Veenker, lote 7k~ft$^2$), la utilidad siempre es negativa porque el costo crece m\'as r\'apido que el precio. Las mejores combinaciones son minimalistas: superficies habitables de 900~ft$^2$, dos dormitorios, un ba\~no completo y sin medios ba\~nos ni segundas cocinas (archivo \texttt{univariate.csv:2-40}). Incrementar dormitorios, ba\~nos o \emph{Overall Qual} por encima de 8 s\'olo agrava la p\'erdida, lo que refuerza la idea de rendimientos decrecientes observada en la secci\'on econ\'omica.

Los barridos bivariados confirman el patr\'on: matrices \texttt{bivariate\_Gr\_Liv\_Area\_\_Overall\_Qual.csv} y \texttt{bivariate\_Gr\_Liv\_Area\_\_Full\_Bath.csv} muestran que la combinaci\'on ``m\'inimo metraje + calidad media'' domina cualquier configuraci\'on de lujo, y \texttt{bivariate\_Bedroom\_AbvGr\_\_Full\_Bath.csv:2-17} evidencia que pasar de dos dormitorios / un ba\~no a configuraciones mayores reduce la utilidad en m\'as de 100~k~USD.

\subsubsection*{Escenarios tipo}
\pgfplotstabletypeset[
  columns={scenario,price_xgb,cost,utility_xgb,roi_xgb_pct},
  columns/scenario/.style={column name=Escenario,string type},
  columns/price_xgb/.style={column name=Precio XGB, sci=false, fixed, precision=0},
  columns/cost/.style={column name=Costo, sci=false, fixed, precision=0},
  columns/utility_xgb/.style={column name=Utilidad XGB, sci=false, fixed, precision=0},
  columns/roi_xgb_pct/.style={column name=ROI XGB (\%), fixed, fixed zerofill, precision=1},
  every head row/.style={before row=\toprule, after row=\midrule},
  every last row/.style={after row=\bottomrule},
  font=\small,
]{\scenarios}

La tabla anterior (datos en \texttt{sensitivity/scenarios.csv}) resume tres dise\~nos de referencia. El modelo XGB confirma que incluso la casa econ\'omica necesita m\'as de 380~k~USD adicionales para ser rentable; la versi\'on premium empeora el ROI hasta -65~\%, de modo que no conviene sobredimensionar calidades mientras los costos unitarios no bajen.

\subsubsection*{An\'alisis de costos (tornado)}
La variaci\'on de costos (\texttt{tornado.csv}) revela que el rubro que m\'as impacta la utilidad es el costo base de construcci\'on (variar \emph{construction\_cost} en \(\pm 20\%\) cambia la utilidad XGB en hasta 101~k~USD), seguido por los coeficientes de dormitorios (46.8~k~USD) y cocina (24~k~USD). Mejorar la eficiencia de mano de obra o negociar materiales en esos rubros es la palanca m\'as efectiva antes de pensar en nuevos atributos.

\section{Construcci\'on: comparaci\'on Regresi\'on vs. XGB}
\subsection{Brecha global por presupuesto}
La Figura~\ref{fig:budgetPrices} muestra que el XGB entrega primas entre 10k y 46k USD sobre la regresi\'on para todos los presupuestos. La diferencia aumenta con el capital porque el XGB captura amenities high-end que la regresi\'on (lineal) suaviza.

\begin{figure}[h]
  \centering
  \begin{tikzpicture}
    \begin{axis}[
      width=0.9\textwidth, height=6cm,
      xlabel={Presupuesto (kUSD)}, ylabel={Precio medio (USD)},
      xmin=350, xmax=1050,
      ymin=350000, ymax=660000,
      ymajorgrids=true,
      legend style={at={(0.02,0.98)},anchor=north west},
      legend cell align=left,
    ]
      \addplot+[mark=*, thick, color=blue] table[x expr=\thisrow{budget}/1000,y=avg_price_xgb]{\budgetdata};
      \addlegendentry{Precio XGB}
      \addplot+[mark=square*, thick, color=orange] table[x expr=\thisrow{budget}/1000,y=avg_price_reg]{\budgetdata};
      \addlegendentry{Precio regresi\'on}
      \addplot+[mark=triangle*, dashed, color=black] table[x expr=\thisrow{budget}/1000,y=avg_delta_abs]{\budgetdata};
      \addlegendentry{Brecha XGB-Reg}
    \end{axis}
  \end{tikzpicture}
  \caption{Precio medio seg\'un presupuesto. El gap XGB--Reg crece con el capital disponible.}
  \label{fig:budgetPrices}
\end{figure}

\subsection{Barrios con mayor uplift}
\begin{figure}[h]
  \centering
  \includegraphics[width=0.9\textwidth]{figures/gap_by_neighborhood.png}
  \caption{Barrios con mayor uplift promedio (XGB vs. Reg). El eje horizontal ordena el gap para facilitar la lectura.}
  \label{fig:gapNeigh}
\end{figure}
Los barrios del quintil superior de precio hist\'orico (SawyerW, Veenker, NoRidge, Blueste) capturan el mayor gap porque el MIP construye garages Ex, acabados Kitchen/Bath Ex o Screen Porch extensos que la regresi\'on lineal no pondera correctamente. Esto evidencia que la regresi\'on sirve como baseline conservadora, pero la decisi\'on de invertir debe basarse en la predicci\'on XGB.

\subsection{Distribuci\'on de brecha porcentual por presupuesto}
\begin{figure}[h]
  \centering
  \includegraphics[width=0.9\textwidth]{figures/delta_pct_by_budget.png}
  \caption{Histograma de brecha porcentual (XGB--Reg) segmentado por presupuesto.}
  \label{fig:deltaPctBudget}
\end{figure}
La superposici\'on permite ver que los presupuestos altos tienden a concentrarse en brechas entre 8~\% y 15~\%, mientras que en 400k aparece mayor dispersi\'on (desde ligeras subestimaciones negativas hasta gap superiores a 20~\%). Esta lectura ayuda a priorizar en qu\'e presupuestos conviene confiar en la regresi\'on como cota inferior y cu\'ando el XGB aporta una prima relevante.

\subsection{Ranking de ROI por barrio}
\pgfplotstabletypeset[
  row predicate/.code={\ifnum\pgfplotstablerow<8\relax\else\pgfplotstableuserowfalse\fi},
  columns={neighborhood,avg_roi,avg_price,avg_delta_abs,n_runs},
  columns/neighborhood/.style={column name=Barrio,string type},
  columns/avg_roi/.style={column name=ROI prom., fixed, fixed zerofill, precision=3},
  columns/avg_price/.style={column name=Precio XGB prom., sci=false, fixed, precision=0},
  columns/avg_delta_abs/.style={column name=Gap XGB-Reg, sci=false, fixed, precision=0},
  columns/n_runs/.style={column name=Corridas},
  every head row/.style={before row=\toprule, after row=\midrule},
  every last row/.style={after row=\bottomrule},
  font=\small,
]{\roitab}

SawyerW y Veenker lideran en ROI (11~\%), alcanzando utilidades medias de 70k USD por casa. Barrios como NPkVill o SWISU muestran ROI similares pero con precios objetivo m\'as bajos, lo que permite calibrar la cartera seg\'un apetito de riesgo.

\section{Construcci\'on: rentabilidad y control de riesgo}
\subsection{Distribuci\'on del ROI}
\begin{figure}[h]
  \centering
  \begin{tikzpicture}
    \begin{axis}[
      width=0.9\textwidth, height=5cm,
      ybar,
      xlabel={ROI (intervalos)}, ylabel={Frecuencia},
      symbolic x coords={-0.20---0.15,-0.15---0.10,-0.10---0.05,-0.05--0.00,0.00--0.05,0.05--0.10,0.10--0.15,0.15--0.20,0.20--0.25,0.25--0.30,0.30--0.35,0.35--0.40},
      xtick=data,
      xticklabel style={rotate=45, font=\scriptsize},
      bar width=8pt,
      ymajorgrids=true,
    ]
      \addplot+[fill=blue!65] table[x=bucket,y=count]{\roihist};
    \end{axis}
  \end{tikzpicture}
  \caption{Distribuci\'on de ROI (723 soluciones). El 75~\% cae entre 0~\% y 14~\%; los extremos cubren [-17.7~\%, 38.3~\%].}
  \label{fig:roiHist}
\end{figure}
La curva deja tres mensajes concretos para el comit\'e: (i) el 75~\% de las corridas cae entre 0~\% y 14~\%, por lo que el negocio es estable pero no explosivo; (ii) la cola izquierda (12.6~\%) aparece cuando el costo m\'inimo exigido es demasiado alto para el barrio y la utilidad se vuelve negativa; y (iii) las colas derechas (>30~\%) son muy puntuales: se dan s\'olo en barrios con slack suficiente para finishes extraordinarios. Esto aterriza la idea de saturaci\'on econ\'omica: subir presupuesto eleva la utilidad absoluta, pero el retorno porcentual converge r\'apido.

\subsection{Escenarios con ROI negativo}
\begin{figure}[h]
  \centering
  \begin{tikzpicture}
    \begin{axis}[
      width=0.7\textwidth, height=4.2cm,
      ybar, ymin=0,
      ylabel={Casos ROI $<0$}, xlabel={Presupuesto},
      symbolic x coords={400000.0,600000.0,800000.0,1000000.0},
      xtick=data,
      nodes near coords,
      ymajorgrids=true,
    ]
      \addplot+[fill=red!70] table[x=budget,y=neg_runs]{\budgetdata};
    \end{axis}
  \end{tikzpicture}
  \caption{Conteo de corridas con utilidad negativa por presupuesto (antes de descartar las infactibilidades de 400k).}
  \label{fig:negatives}
\end{figure}
Los casos negativos se concentran en el presupuesto de 400k (21~\% de esas corridas) porque no alcanza para financiar la calidad m\'inima exigida en algunos barrios. Con capitales de 600k--1M el porcentaje cae bajo 10~\%, lo que respalda usar el benchmark como filtro previo para reuniones con el comit\'e de inversi\'on.

\subsection{M\'etricas completas por presupuesto}
\pgfplotstabletypeset[
  columns={budget,avg_price_xgb,avg_price_reg,avg_delta_abs,avg_delta_pct,avg_slack,avg_obj,avg_roi_pct,avg_budget_used_pct,neg_runs,n_runs},
  columns/budget/.style={column name=Presupuesto, fixed, precision=0},
  columns/avg_price_xgb/.style={column name=Precio XGB, sci=false, fixed, precision=0},
  columns/avg_price_reg/.style={column name=Precio Reg, sci=false, fixed, precision=0},
  columns/avg_delta_abs/.style={column name=Gap USD, sci=false, fixed, precision=0},
  columns/avg_delta_pct/.style={column name=Uplift (\%), fixed, precision=2},
  columns/avg_slack/.style={column name=Slack, sci=false, fixed, precision=0},
  columns/avg_obj/.style={column name=Utilidad, sci=false, fixed, precision=0},
  columns/avg_roi_pct/.style={column name=ROI (\%), fixed, precision=1},
  columns/avg_budget_used_pct/.style={column name=Presupuesto usado (\%), fixed, precision=1},
  columns/neg_runs/.style={column name=ROI$<0$},
  columns/n_runs/.style={column name=Corridas},
  every head row/.style={before row=\toprule, after row=\midrule},
  every last row/.style={after row=\bottomrule},
  font=\small,
]{\budgetdata}

Esta tabla replica los KPIs definidos en el Informe~2: el \emph{Porcentaje neto de mejora} se reporta como ``Uplift (\%)'', mientras que ``Presupuesto usado (\%)'' cuantifica el uso efectivo de capital. Se observa el mismo patr\'on que en la entrega anterior: el tier de 400k usa apenas 91~\% del presupuesto y entrega el mayor ROI porcentual (11~\%), mientras que los tiers altos consumen menos del 65~\% del capital disponible y, aunque elevan la utilidad absoluta, diluyen el retorno relativo. Esto valida la hip\'otesis de rendimientos decrecientes y de sobreinversi\'on que quer\'iamos evidenciar para la siguiente entrega.

\section{Construcci\'on: sensibilidad al tama\~no de lote}
\begin{figure}[h]
  \centering
  \begin{tikzpicture}
    \begin{axis}[
      width=0.9\textwidth, height=5cm,
      xlabel={Lot Area (ft$^2$)}, ylabel={Precio XGB promedio},
      ymajorgrids=true,
    ]
      \addplot+[mark=*, color=blue] table[x=lot_area,y=avg_price_xgb]{\lotdata};
    \end{axis}
  \end{tikzpicture}
  \caption{Precio promedio vs. superficie de lote.}
  \label{fig:lot}
\end{figure}
La curva es casi plana incluso tras a\~nadir los extremos (1\,300 y 31\,700~ft$^2$). Esto respalda la hip\'otesis de saturaci\'on: una vez aseguradas calidades altas, el valor marginal del terreno es peque\~no. Para explotar lotes grandes habr\'ia que habilitar amenities espec\'ificas (pool, ADU, paisajismo) o segmentos con precios sombra diferenciados.

\section{Construcci\'on: sensibilidad MIP por escenarios}
Usando \texttt{run\_mip\_sensitivity.py} se definieron 10 escenarios “tipo” (starter, move-up, premium, compact, etc.) con combinaciones de presupuesto y restricciones (\( \geq \) dormitorios, ba\~nos, calidad, superficie). Cada corrida guarda el tag del escenario en \texttt{mip\_scenarios.csv}; el script \texttt{process\_mip\_scenarios.py} produce tabulados resumidos (\texttt{mip\_scenarios\_summary.csv} y \texttt{mip\_scenarios\_roi\_by\_budget.csv}) usados aqu\'i.

\paragraph{Definici\'on de escenarios}
\begin{description}
  \item[\textbf{Starter 1Fam}] Presupuestos 350--450k USD, vivienda unifamiliar b\'asica con al menos 2 dormitorios, 1 ba\~no completo, calidad \( \geq 6 \) y \(\geq\)900~ft$^2$ de \emph{Gr Liv Area}.
  \item[\textbf{Move-up 1Fam}] 500--700k USD, orientado a familias que suben de segmento: 3 dormitorios, 2 ba\~nos, calidad \( \geq 7 \) y 1\,300~ft$^2$ m\'inimos.
  \item[\textbf{Premium 1Fam}] 800k--1M USD, 4 dormitorios, 3 ba\~nos completos, 2 cocinas, calidad \( \geq 9 \), al menos 2\,000~ft$^2$ y garages de \(\geq\)500~ft$^2$.
  \item[\textbf{Compact Duplex HighQual}] 500--600k USD, duplex premium pero acotado: 3 dormitorios, 2 ba\~nos, calidad \( \geq 8 \) y l\'imite de 1\,600~ft$^2$ para mantener el perfil compacto.
  \item[\textbf{Premium Duplex}] 650--850k USD, duplex amplio con 4 dormitorios, 3 ba\~nos, 2 cocinas y garages \(\geq\)400~ft$^2$.
  \item[\textbf{Townhouse Mid}] 450--650k USD, \emph{townhouse} en mediana gama (TwnhsE) con 3 dormitorios, 2 ba\~nos y l\'imite de 1\,700~ft$^2$.
  \item[\textbf{Investor Low}] 300--400k USD, mezcla 1Fam/Duplex pensada para inversionistas frugales: 2 dormitorios, 1 ba\~no, \emph{Gr Liv Area} \(\leq\)1\,200~ft$^2$ y calidad \( \geq 6 \).
  \item[\textbf{High Density Mix}] 600--700k USD, combina 1Fam/Duplex/Townhouse con perfiles ``balanced''/``bound'' y exige 3 dormitorios, 2 ba\~nos y calidad \( \geq 8 \).
  \item[\textbf{Min Baths vs Beds}] 500--650k USD, 1Fam con foco en confort: 4 dormitorios, 3 ba\~nos completos y al menos un medio ba\~no.
  \item[\textbf{Garage Focus}] 550--750k USD, mezcla 1Fam/Duplex con garages entre 400 y 700~ft$^2$, 3 dormitorios y 2 ba\~nos.
\end{description}
Cuando hablamos de \emph{escenarios compactos} nos referimos a Starter, Townhouse Mid, Investor Low y Compact Duplex HighQual; todos limitan la superficie habitable (\(\leq\)1\,700~ft$^2$) y operan con presupuestos \(\leq\)600k USD, de ah\'i que mantengan costos contenidos.

\begin{table}[h]
  \centering
  \caption{Escenarios con mayor ROI (datos de \texttt{mip\_scenarios\_summary.csv}).}
  \label{tab:mip-scenarios}
  \footnotesize
  \begin{tabular}{lccc rr}
    \toprule
    Escenario & Budget min & Budget max & Mejor budget & ROI prom. (\%) & ROI max. (\%) \\
    \midrule
    high\_density\_mix & 600k & 700k & 600k & 0.95 & 5.88 \\
    townhouse\_mid & 550k & 650k & 650k & 1.18 & 5.06 \\
    moveup\_1fam & 500k & 700k & 700k & -2.18 & 3.26 \\
    garage\_focus & 550k & 750k & 650k & 0.28 & 3.25 \\
    compact\_duplex\_highqual & 500k & 600k & 500k & 1.56 & 1.60 \\
    min\_baths\_vs\_beds & 500k & 650k & 650k & -11.12 & -1.51 \\
    \bottomrule
  \end{tabular}
\end{table}

La tabla anterior muestra que s\'olo los escenarios compactos (Starter, Townhouse Mid, Investor Low y Compact Duplex) se acercan a ROI positivos, y que los premium requieren >800~k~USD para sostener las calidades exigidas. Aun as\'i, todos los escenarios mantienen slack elevados (200–400~k USD), lo que confirma que el presupuesto no es el cuello de botella sino los costos unitarios.

El ROI promedio es menos negativo en estas variantes compactas que en los premium, porque limitan la superficie y mantienen costos contenidos. Aun as\'i, incluso esas combinaciones siguen mostrando retornos negativos, lo que refuerza que el cuello de botella es el costo unitario y no el presupuesto.

\begin{figure}[h]
  \centering
  \includegraphics[width=0.85\textwidth]{figures/mip_roi_vs_budget.png}
  \caption{ROI (\%) por presupuesto en escenarios representativos (fuente: \texttt{mip\_roi\_vs\_budget.png}).}
  \label{fig:mipRoiBudget}
\end{figure}

El gr\'afico confirma rendimientos decrecientes: aumentar el budget mejora la utilidad absoluta (menor slack) pero no revierte la p\'erdida porcentual. Las curvas se aplanan sobre -50~\%, por lo que las inversiones de alto presupuesto s\'olo tienen sentido si se introducen ingresos adicionales (amenities o subsidios). Los escenarios premium etiquetados como ``A'' y ``B'' son los menos sensibles gracias a restricciones compartidas, pero su ROI sigue anclado en torno a -55~\%.

\section{Construcci\'on: perfil constructivo vs. base hist\'orica}
\begin{figure}[h]
  \centering
  \begin{tikzpicture}
    \begin{groupplot}[
      group style={group size=3 by 2, horizontal sep=1.1cm, vertical sep=1.1cm},
      ymin=0,
      ylabel style={font=\small},
      xlabel style={font=\small},
      ymajorgrids=true,
      width=0.32\textwidth,
      height=4.5cm,
    ]
      \nextgroupplot[title={Gr Liv Area}]
        \addplot+[ybar, fill=blue!60, bar width=6pt, xtick=data,
          xticklabels from table={\featuredata}{neighborhood},
          xticklabel style={rotate=45, font=\scriptsize}]
          table[x expr=\coordindex, y=bench_gr_liv_area]{\featuredata};
        \addplot+[ybar, fill=orange!70, bar width=6pt] table[x expr=\coordindex, y=base_gr_liv_area]{\featuredata};
        \legend{Benchmark,Base}
      \nextgroupplot[title={Total Bsmt}, xticklabels=\empty]
        \addplot+[ybar, fill=blue!60, bar width=6pt] table[x expr=\coordindex, y=bench_bsmt]{\featuredata};
        \addplot+[ybar, fill=orange!70, bar width=6pt] table[x expr=\coordindex, y=base_bsmt]{\featuredata};
      \nextgroupplot[title={Garage Area}, xticklabels=\empty]
        \addplot+[ybar, fill=blue!60, bar width=6pt] table[x expr=\coordindex, y=bench_garage_area]{\featuredata};
        \addplot+[ybar, fill=orange!70, bar width=6pt] table[x expr=\coordindex, y=base_garage_area]{\featuredata};
      \nextgroupplot[title={Dormitorios}]
        \addplot+[ybar, fill=blue!60, bar width=6pt, xtick=data,
          xticklabels from table={\featuredata}{neighborhood},
          xticklabel style={rotate=45, font=\scriptsize}]
          table[x expr=\coordindex, y=bench_beds]{\featuredata};
        \addplot+[ybar, fill=orange!70, bar width=6pt] table[x expr=\coordindex, y=base_beds]{\featuredata};
      \nextgroupplot[title={Baths}, xticklabels=\empty]
        \addplot+[ybar, fill=blue!60, bar width=6pt] table[x expr=\coordindex, y=bench_fullbath]{\featuredata};
        \addplot+[ybar, fill=orange!70, bar width=6pt] table[x expr=\coordindex, y=base_fullbath]{\featuredata};
      \nextgroupplot[title={Overall Qual}, xticklabels=\empty]
        \addplot+[ybar, fill=blue!60, bar width=6pt] table[x expr=\coordindex, y=bench_overall]{\featuredata};
        \addplot+[ybar, fill=orange!70, bar width=6pt] table[x expr=\coordindex, y=base_overall]{\featuredata};
    \end{groupplot}
  \end{tikzpicture}
  \caption{Comparaci\'on de atributos promedio entre el benchmark y la base hist\'orica.}
  \label{fig:features}
\end{figure}
El MIP prioriza aumentar \emph{Gr Liv Area}, s\'otano y garage hasta el l\'imite permitido del barrio, mientras que dormitorios y ba\~nos se mantienen cercanos a la media. La columna ``Overall Qual'' evidencia que siempre se trabaja dentro del rango autorizado (Avg--Ex) pero inclin\'andose sistem\'aticamente hacia calidades sobresalientes, lo que explica la brecha con la regresi\'on y las utilidades observadas.

\section{Construcci\'on: viabilidad e infactibilidades}
Los 61 escenarios infactibles comparten un rasgo: presupuesto de 400k en barrios/lotes donde el costo base de cimientos, garage y acabados supera ese techo. El benchmark deja trazados esos casos en \texttt{missing\_combos.csv} para evitar invertir tiempo de dise\~no comercial en oportunidades inviables. Para presupuestos \ensuremath{\geq}600k no se observaron infactibilidades.

\section{Conclusiones y siguientes pasos}
\begin{itemize}
  \item La regresi\'on lineal subestima 5--15~\% del precio realista post-obra. El XGB debe ser la referencia comercial y la regresi\'on queda como cota conservadora.
  \item El ROI medio (8.6~\%) muestra rendimientos decrecientes: conviene priorizar presupuestos 400k--600k para eficiencia y usar 800k--1M cuando se busquen slack altos para amenidades especiales.
  \item El modelo ya detecta autom\'aticamente escenarios con ROI negativo o incluso infactibles, evitando gastos en planos o marketing para proyectos perdedores.
  \item Existe espacio para ampliar el estudio incorporando presupuestos intermedios (500k, 700k), percentiles de mercado extremos (p10/p90) y objetivos que valoricen lotes grandes (paisajismo o ADU). Adem\'as, se puede exigir un ROI m\'inimo por barrio para generar portafolios m\'as conservadores.
\end{itemize}

\end{document}
