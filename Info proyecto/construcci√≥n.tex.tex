\documentclass{article}
\usepackage{graphicx} % Required for inserting images

\title{Construcción}
\author{Grupo 5 Capstone}
\date{October 2025}

\begin{document}

\maketitle

\[
\min_{F}\; Z(F)
= \frac{1}{2}\sum_{a\in A}
\left[
  \int_{0}^{f_a} c_a(x,f_{a'})\,dx
  + 
  \int_{0}^{f_a} c_a(x,0)\,dx
\right].
\]
\section{Introduction}
\section{Supuestos}
1. Se asume que cada vivienda se construye desde cero, sobre un terreno definido cuyas características físicas son parámetros fijos y no pueden modificarse. El modelo busca maximizar la rentabilidad de la construcción, definida como la diferencia entre la tasación post-construcción y el costo total de edificación, sujeto a las restricciones físicas, estructurales y de consistencia espacial de la vivienda.\\
\\ 
2. Todas las construcciones corresponden a nueva edificación, por lo que se asume que la calidad y condición de todos los elementos estructurales es excelente:

OverallQual = 10, OverallCond = 10

ExterQual = Ex, ExterCond = Ex

BsmtQual = Ex, HeatingQC = Ex

KitchenQual = Ex

Esto implica que no existen depreciaciones asociadas a desgaste, y que los costos reflejan materiales y terminaciones de alta gama.
\\
\\
3. Solo puede seleccionarse una categoría por variable estructural o de sistema, tales como:

Foundation, RoofStyle, RoofMatl, Heating, Electrical, GarageType, Exterior1st, HouseStyle, entre otras.

Esta exclusividad asegura consistencia constructiva y evita configuraciones físicamente imposibles (por ejemplo, combinar techos de distintos materiales o cimentaciones incompatibles).\\
\\
4. Todas las áreas modeladas corresponden a espacios terminados; no existen sectores sin finalizar.
Se imponen las siguientes condiciones:

BsmtUnfSF = 0, LowQualFinSF = 0.

Los tipos de sótano Unf o NA implican área de acabado nula.

Los estilos de vivienda con niveles “parciales” (1.5Fin, 2.5Fin, etc.) son excluidos.

Así, cada espacio construido contribuye a la superficie útil total y tiene costo de terminación asociado.\\
\\
5. Las viviendas pueden tener uno o dos pisos completos, sin niveles medios ni áticos.\\

El área de cimentación y la huella de construcción equivalen al área del primer piso ($AreaFoundation_i = 1stFlrSF_i$).
Las áreas asociadas al segundo piso (baños, cocinas, dormitorios, otros) se activan solo si Floor2_i = 1, con cotas máximas realistas.\\
Estas cotas derivan de proporciones típicas de distribución de superficie habitacional según U.S. DOE (2024).\\
\\
6. El tipo de cimentación (Foundation) define la base estructural del edificio:

Si la vivienda tiene sótano, se permiten solo BrkTil, CBlock, PConc, o Stone.

Si no tiene sótano, se permiten Slab o Wood.

El área de cimentación se iguala al área del primer piso y su costo depende del tipo seleccionado.\\

7. Se considera al menos una cocina y un baño completo en el primer piso, con límites superiores de repetición por tipo de vivienda.\\

En viviendas tipo Duplex o Two-Family Conversion, se permite la replicación de ambientes equivalentes en el segundo piso (cocina, baño, dormitorios).\\
\\
8. El área real del techo se calcula como
ActualRoofArea_i = Σ_s Σ_m γ_{s,m}·Z_{i,s,m},
donde γ_{s,m} es un factor de pendiente (1.1–1.4).
\section{Función Objetivo}
\begin{center}
    $\max \; \Pi = V_{i}^{post} - C^{Total}$
\end{center}
Donde:\\
\begin{itemize}
    \item $\Pi$= Rentabilidad
    \item $V_{i}^{post}=$Valor de la casa construida.
    \item $C_{i}^{Total}=$Costo Total construcción
\end{itemize}
$C_{i}^{Total}=C_{i}^{Foundation}+C_i^{Roof}+C_{i}^{Heating}+C_{i}^{CentralAir}+C_{i}^{Electrical}+C_{i}^{PavedDrive}+C_{i}^{Kitchen}+C_{i}^{HalfBaths}+C_{i}^{FullBaths}+C_{i}^{Bedroom}+C_{i}^{Garage}+C_{i}^{Porch}+C_{i}^{WoodDeck}+C_{i}^{Reja} + C{i}^{Basement}+C_{i}^{MasVnr}+C_{i}^{Exterior}+C_{i}^{MiscFeature} +C_{i}^{FirePlaces}$\\
\\
\begin{itemize}
    \item $C_{i}^{Foundation}=\sum _{f} L_{i,f}\cdot C_{f}$
    \item $C_i^{Roof}= \sum_{s} \sum_{m} C_{m}\cdot (\gamma_{s,m}\cdot Z_{i,s,m})$
    \item $C_{i}^{Heating}=\sum_{h} C_{h,Ex}\cdot HasHeating_{i,h}$
    \item $C_{i}^{CentralAir}= \sum_{a} C_{i,a}$
    \item $C_{i}^{Electrical}=\sum_{e} C_{e} \cdot Electrical_{i,e}$
    \item $C_{i}^{PavedDrive}= \sum_{d} C_{d}\cdot PavedDrive_{i,d}$
    \item $C_{i}^{Kitchen}=\sum_{k}C_{k,Ex}\cdot AreaKitchen_{i}$
    \item $C_{i}^{HalfBaths}= C_{HalfBath}\cdot AreaHalfBath_{i}$
    \item $C_{i}^{FullBaths}= C_{FullBath}\cdot AreaFullBath_{i}$
    \item $C_{i}^{Bedroom}= C_{Bedroom}\cdot AreaBedroom_{i}$
    \item $C_{i}^{Garage}=C_{Garage}\cdot GarageArea_{i}$
    \item $C_{i}^{Porch}= OpenPorch_{i}\cdot C_{OpenPorch}+EnclosedPorch_{i}\cdot C_{EnclosedPorch}+3SsnPorch_{i}\cdot C_{3SsnPorch}+ScreenPorch_{i} \cdot C_{ScreenPorch}$
    \item $C_{i}^{WoodDeck}=WoodDesk\cdot C_{WoodDeck}$
    \item $C_{i}^{Fence}=L_{i}\cdot C_{Fence}$
    \item $C_{i}^{Exterior}=C_{e1} \cdot AreaExterior1st_{i,e1}$
    \item $C_i^{MasVnr}=\sum _{t}MvProd_{i,t}\cdot C_{t}$
    \item $C_{i}^{Garage}= \sum_{g} GA_{i,g}\cdot C_{garage}$
    \item $C_{i}^{FirePlaces}=C_{f,Ex}\cdot FirePlaces_{i}$
    \item $C_{i}^{Basement}=$
\end{itemize}

\section{Variables Binarias:}
\begin{itemize}

    \item $Floor1_{i} \in 0,1 \hspace{5mm} \forall i \in \mathcal{I}$: Casa i tiene 1 piso
    \item $Floor2_{i} \in 0,1 \hspace{5mm} \forall i \in \mathcal{I}$: Casa tiene 2 pisos
    \item $SameMaterial_{i} \in 0,1$
    \item $HasReja_{i} \in 0,1$

\end{itemize}
\section{Variables de área}
\begin{itemize}
    \item $AreaBedrooms_{i} \in \mathbf{Z}_{\geq 0}$
    \item $AreaBedroom1_{i}$
    \item $AreaBedroom2_{i}$
    \item $AreaMasterBedroom_{i}$
    \item $AreaOtherRooms1_{i}$
    \item $AreaOtherRooms2_{i}$
    \item $AreaOther_{i}$
    \item $AreaKitchen_{i} \in \mathbf{Z}_{\geq 0}$
    \item $AreaFullBath_{i} \in \mathbf{Z}_{\geq 0}$
    \item $AreaHalfBath_{i} \in \mathbf{Z}_{\geq 0}$
    \item $AreaKitchen1_{i} \in \mathbf{Z}_{\geq 0}$
    \item $AreaKitchen2_{i}\in \mathbf{Z}_{\geq 0}$
    \item $AreaFullBath1_{i}\in \mathbf{Z}_{\geq 0}$
    \item $AreaFullBath2_{i}\in \mathbf{Z}_{\geq 0}$
    \item $AreaHalfBath1_{i}\in \mathbf{Z}_{\geq 0}$
    \item $AreaHalfBath2_{i}\in \mathbf{Z}_{\geq 0}$
    \item $AreaFoundation_{i}\in \mathbf{Z}_{\geq 0}$
    \item $AreaRoof_{i} \in \mathbf{Z}_{\geq 0}$
    \item $PR1_i,\, PR2_i \ge 0$ \quad (\textit{auxiliares para linealizar $AreaRoof_i$})
    \item $PR1_{i}:$ Área del primer piso que se cubre con techo, si vivienda tiene 1 piso
    \item $PR2_{i}$ Area del segundo piso que se cubre con techo, si vivienda tiene 2 pisos4
    \item $P_{i}^{(1)}$
    \item $P_{i}^{(2)}$
    \item $W_{i,e1}:$ cuanta área exterior de la casa se cubre con el material e1
    \item $GA_{i,g}$: área del garage tipo g
\end{itemize}

\section{Variables de conteo}
\begin{itemize}
    \item $FullBath1_{i}\in \mathbf{Z}$ cantidad de Full Baths en primer piso
    \item $FullBath2_{i}$ cantidad de FullBath en segundo piso
    \item $HalfBath1_{i} \in \mathbf{Z}_{\geq 0}$
    \item $HalfBath2_{i} \in \mathbf{Z}_{\geq 0}$
    \item $Kitchen1_{i} \in \mathbf{Z}_{\geq 0}$
    \item $Kitchen2_{i} \in \mathbf{Z}_{\geq 0}$
    \item $Bedroom1_{i}$
    \item $Bedroom2_{i}$
    \item $OtherRooms1_{i}$
    \item $OtherRooms2_{i}$
    \item $OtherRooms_{i}$
\end{itemize}

\section{Restricciones}
\subsubsection{Restricciones de exclusividad}
\begin{align}
    &\sum_{s} MSSubClass_{i,s}=1 \hspace{5mm} \forall i \in \mathcal{I} \\
    \\
    & \sum_{\mathcal{B}\in b} BldgType_{i,b}= 1 \hspace{5mm} \forall i \in \mathcal{I}\\
    \\
    & \sum _{hs} HouseStyle_{i,hs}=1\\
    \\
    &\sum_{r} RoofStyle_{i,r}=1\\
    \\
    &\sum_{m} RoofMatl_{i,m}=1\\
    \\
    &\sum_{e1} Exterior1st_{i,e1}=1,\\
    \\
    &\sum_{e2} Exterior2nd_{i,e2}=1\\
    \\
    &\sum_{t} MasVnrType_{i,t}=1\\
    \\
    &\sum_{f} Foundation_{i,f}=1\\
    \\
    &\sum_{x} BsmtExpoure_{i,x}=1\\
    \\
    &\sum_{b1} BsmtFinType1_{i,b1}=1\\
    \\
    &\sum_{b2} BsmtFinType2_{i,b2}=1\\
    \\
    &\sum_{h} Heating_{i,h}= 1\\
    \\
    &\sum_{a} CentralAir_{i,a}=1\\
    \\
    &\sum_{e} Electrical_{i,e}=1\\
    \\
    &\sum_{g} GarageType_{i,g}=1\\
    \\
    &\sum_{gf} GarageFinish_{i,gf}=1\\
    \\
    &\sum_{p} PavedDrive_{i,p}=1\\
    \\
    &\sum_{misc} MiscFeature_{i,misc}=1\\
    
    
\end{align}

\subsection{Consistencia de Áreas}
\subsubsection{Áreas construidas no pueden sobrepasar el área del terreno}\\
\begin{align}
    &1stFlrSF_{i} + TotalPorchSF_{i} + AreaPool_{i} \leq LotArea_{i}, \hspace{5mm}\forall i \in \mathcal{I} \\
\end{align}
\begin{align}
& \boxed{LotOc_i \;=\; 1stFlrSF_i \;+\; TotalPorchSF_i \;+\; WoodDeckSF_i \;+\; AreaPool_i \;+\; GarageArea_i \hspace {5mm} \forall i \in \mathcal{I}}
\end{align}
\subsubsection{El segundo piso no puede ser más grande que el primero}
\begin{align}
    
&2ndFlrSF_{i} \leq 1stFlrSF_{i} \hspace{5mm}\forall i \in \mathcal{I}
\end{align}
\subsubsection{Area Habitable}
\begin{align}
    &GrLivArea = 1stFlrSF_{i} + 2ndFlrSF_{i}
\end{align}
\subsubsection{Area Total FullBath  es igual al area baños 1er piso + area baños 2do piso}
\begin{align}
    &AreaFullBath_{i}=AreaFullBath1_{i}+ AreaFullBath2_{i}
\end{align}
\subsubsection{Area Total HalfBath es igual al area baños 1er piso + Area baños 2do piso}
\begin{align}
    &AreaHalfBath_{i}=AreaHalfBath1_{i}+AreaHalfBath2_{i}
\end{align}
\subsubsection{Tiene que haber un baño en el primer piso}
\begin{align}
    FullBath1_{i}\geq 1
\end{align}

\subsubsection{Tiene que haber una cocina en el primer piso}
\begin{align}
    Kitchen1_{i}\geq 1
\end{align}

\subsubsection{Areas Primer y segundo piso}
\begin{align}
    &2ndFlrSF_{i}\leq M_{max}^{2ndFlrSF} \cdot Floor2_{i}\\
    &2ndFlrSF_{i}\geq \epsilon \cdot Floor2_{i}\\
    &1stFlrSF_{i}\geq \epsilon´\cdot (Floor1_{i}+Floor2_{i})
\end{align}\\
Donde:
\begin{itemize}
    \item $\epsilon=450$
    \item $\epsilon ´=350$
\end{itemize}
\subsection{Consistencia Cantidades}
\subsubsection{Cantidad total de FullBaths}
    \[
FullBath_i \;=\; FullBath1_i + FullBath2_i \hspace{8mm} \forall i.
\]
\subsubsection{Cantidad Total de HalfBath}
    \[
HalfBath_i \;=\; HalfBath1_i + HalfBath2_i \hspace{8mm} \forall i.
\]

\subsubsection{Cantidad Total de Cocinas}
\[
Kitchen_i \;=\; Kitchen1_i + Kitchen2_i \hspace{8mm} \forall i.
\]

\subsection{Máximo de repeticiones}
\begin{itemize}
    \item Parámetros para Bedrooms según tipo de vivienda
    \begin{itemize}
        \item $Bed_{max}^{1Fam}=6$\\
        \item $Bed_{max}^{TwnhsE}=4$\\
        \item $Bed_{max}^{TwnhsI}=4$\\
        \item $Bed_{max}^{Dplx}=5$\\
        \item $Bed_{max}^{2FmCon}=8$
    \end{itemize}
    \subsubsection{Máxima cantidad de habitaciones:}
    \begin{align}
        Bedrooms_{i}\leq \sum_{b \in \mathcal{B}}Bed_{max}^{b} \cdot BldgType_{i,b} \hspace{5mm } \forall i \in \mathcal{I}
    \end{align}
    \item  Parametros para FullBaths:
    \begin{itemize}
        \item $F_{max}^{1Fam}=4$\\
        \item $F_{max}^{TwnhsE}=3$\\
        \item $F_{max}^{TwnhsI}=3$\\
        \item $F_{max}^{Dplx}=4$\\
        \item $F_{max}^{2FmCon}=6$
    \end{itemize}
    \subsubsection{Máxima cantidad de FullBaths:}
    \begin{align}
        FullBath_{i} \leq \sum_{b} F_{max}^{b} \cdot BldgType_{i,b}
    \end{align}
    \item Parámetros HalfBath:
    \begin{itemize}
        \item $H_{max}^{1Fam}=2$\\
        \item $H_{max}^{TwnhsE}=2$\\
        \item $H_{max}^{TwnhsI}=2$\\
        \item $H_{max}^{Dplx}=2$\\
        \item $H_{max}^{2FmCon}=3$
    \end{itemize}
    \subsubsection{Máxima cantidad de HalfBaths:}
    \begin{align}
        HalfBath_{i} \leq \sum_{b} H_{max}^{b} \cdot BldgType_{i,b}
    \end{align}
    \item Parámetros Cocina:
    \begin{itemize}
        \item $K_{max}^{1Fam}=1$\\
        \item $K_{max}^{TwnhsE}=1$\\
        \item $K_{max}^{TwnhsI}=1$\\
        \item $K_{max}^{Dplx}=2$\\
        \item $K_{max}^{2FmCon}=2$
    \end{itemize}
    \subsubsection{Máxima cantidad de Cocinas:}
    \begin{align}
        Kitchen_{i}\leq \sum _{b} K_{max}^{b} BldgType_{i,b}
    \end{align}
    \item Parámetros Chimenea:
    \begin{itemize}
        \item $Ch_{max}^{1Fam}=1$\\
        \item $Ch_{max}^{TwnhsE}=1$\\
        \item $Ch_{max}^{TwnhsI}=1$\\
        \item $Ch_{max}^{Dplx}=1$\\
        \item $Ch_{max}^{2FmCon}=2$ 
    \end{itemize}
    \subsubsection{Máxima cantidad de chimeneas:}
    \begin{align}
        FirePlaces_{i} \leq \sum_{b} Ch_{max}^{b} BldgType_{i,b}
    \end{align}
\end{itemize}
\subsection{Casa solo puedo tener 1 ó 2 pisos}
\begin{align}
    &Floor1_{i} +Floor2_{i}=1 \hspace{5mm} \forall i \in \mathcal{I}
\end{align}
\subsection{Garage}
\subsubsection{Consistencia Areas Garage}
\begin{align}
    150\cdot GarageCars_{i} \leq GarageArea_{i}\leq 250 \cdot GarageCars_{i}
\end{align}
\subsubsection{Existencias de activación}
\begin{align}
& GarageCars_i \le \overline{C}^{\text{cars}} \,\big(1 - GarageType_{i,NA}\big) \qquad \forall i,\\
& GarageArea_i \le \overline{A}^{\text{garage}}_i \,\big(1 - GarageType_{i,NA}\big) \qquad \forall i.
\end{align}
Donde:
\begin{itemize}
    \item $\bar{C}_{i}^{cars}=4$
    \item $\bar{A}_{i}^{garage}=0.2 LotArea_{i}$
\end{itemize}
\subsubsection{Acabados}
\begin{align}
& GarageFinish_{i,NA} = GarageType_{i,NA} \qquad \forall i,\\
& GarageFinish_{i,\text{Fin}} + GarageFinish_{i,\text{RFn}} 
  = 1 - GarageType_{i,NA} \qquad \forall i.
\end{align}


\subsubsection{Mínimo Funcional}
\begin{align}
& GarageCars_i \ge 1 - GarageType_{i,NA} \qquad \forall i.
\end{align}
\subsection{Cerca}
\begin{itemize}
    \item Parámetros:
    \begin{itemize}
        \item $L^{Reja}_{i}=LotFrontage_{i}$
    \end{itemize}
\end{itemize}
\begin{align}
    C_{i}^{Reja}=
\end{align}

\subsection{Area Techo}

% Techo en planta: AreaRoof_i = 1stFlrSF_i*Floor1_i + 2ndFlrSF_i*Floor2_i
% con auxiliares PR1_i, PR2_i:
\subsubsection{Area techo}
\begin{align}
    &PR1_{i} \leq 1stFloorSF_{i}\\
    &PR1_{i} \leq U_{i}^{(1)}\cdot Floor1_{i}\\
    &PR1_{i} \geq 1stFloorSF_{i}-U_{i}^{(1)}\cdot (1-Floor_{i})\\
    &PR2_{i} \leq 2ndFloorSF_{i},\\
    &PR2_{i} \leq U_{i}^{(2)}\cdot Floor2_{i}\\
    &PR2_{i} \geq 2ndFloorSF_{i}-U_{i}^{(2)}\cdot (1-Floor2_{i})\\
\end{align}
\\
Donde:
\begin{itemize}
    \item $U_{i}^{(1)}\geq 1stFlrSF_{i}$
    \item $U_{i}^{(2)}\geq 2ndFlrSF_{i}$
    \item $U_{i}^{plan} \geq max\{1stFlrSF_{i},2ndFlrSF_{i}\}$: cota superior
\end{itemize}
\\
\subsubsection{Área que debe ser cubierta}\\
\begin{align}
    PlanRoofArea_{i}=PR1_{i}+PR2_{i}\\
\end{align}
\subsubsection{Área real de techo que se debe construir de acuerdo a pendiente del tipo de techo}\\
\begin{align}
    ActualRoofArea_{i}=\sum_{s}\sum_{m} \gamma_{s,m} \cdot Z_{i,s,m}\\
\end{align}
\\
Donde:
\begin{itemize}
    \item $\gamma_{s,m} \geq 1$. Factor de expansión según estilo
    \item  $Z_{i,s,m}$ variable auxiliar que linealiza 
    \item $Y_{i,s,m} \in 0,1$ toma valor 1 cuando se selecciona el estilo s y material m de la casa i
\end{itemize}


\subsubsection{Combinacion estilo y material:Exclusividad}
%-------------------------------
% Conjunción estilo–material (AND lógico)
%-------------------------------
\begin{align}
& Y_{i,s,m} \le RoofStyle_{i,s} \qquad \forall s\in S,\ \forall m\in M, \\[2pt]
& Y_{i,s,m} \le RoofMatl_{i,m} \qquad \forall s\in S,\ \forall m\in M, \\[2pt]
& Y_{i,s,m} \ge RoofStyle_{i,s} + RoofMatl_{i,m} - 1 \qquad \forall s\in S,\ \forall m\in M, \\[4pt]
& \sum_{s}\sum_{m} Y_{i,s,m} = 1.
\end{align}
\subsubsection{Restricciones lineales de techo}
%-------------------------------
% Área real de techo (opcional) y su linealización
%-------------------------------
% Z_{i,s,m} ≈ PlanRoofArea_i * Y_{i,s,m}
\begin{align}
& Z_{i,s,m} \le PlanRoofArea_i \qquad \forall s,m, \\[2pt]
& Z_{i,s,m} \le U^{\text{plan}}_i \, Y_{i,s,m} \qquad \forall s,m, \\[2pt]
& Z_{i,s,m} \ge PlanRoofArea_i - U^{\text{plan}}_i \,(1 - Y_{i,s,m}) \qquad \forall s,m, \\[2pt]
& Z_{i,s,m} \ge 0 \qquad \forall s,m, \\[6pt]



%-----------------------------------------------------
% Notas:
% - Si decides NO usar área real (pendiente/solapes), omite \eqref{eq:actual-roof}
%   y las restricciones de Z; trabaja solo con PlanRoofArea_i (o AreaRoof_i).
% - Las cotas U^{(1)}_i, U^{(2)}_i, U^{plan}_i están alineadas con tus límites
%   globales (p.ej. 1stFlrSF_i ≤ 0.6 LotArea_i, etc.).
% - Si prefieres dependencia solo por estilo (sin material), reemplaza γ_{s,m} por α_s
%   y elimina el índice m en Y y Z (mecánica análoga).
%-----------------------------------------------------

\subsection{Consistencias Áreas Globales}
\begin{align}

& TotalBsmtSF_i = BsmtFinSF1_i + BsmtFinSF2_i,\\
& TotalArea_i = 1stFlrSF_i + 2ndFlrSF_i + TotalBsmtSF_i
\end{align}

\subsection{Límites de ocupación}
\begin{align}
& 1stFlrSF_i \le M_{max}^{1stFlrSF},\quad \\2ndFlrSF_i \le M_{max}^{2ndFlrSF},\\
& TotalBsmtSF_i \le M_{max}^{TotalBasmt},\\ \quad 
& GarageArea_i \le M_{max}^{GarageArea}
\end{align}
\\
Donde:
\begin{itemize}
    \item $M_{max}^{1stFlrSF}= 0.6LotArea$
    \item $M_{max}^{2ndFlrSF}=0.5 LotArea$
    \item $M_{max}^{TotalBasmt}= 0.5LotArea$
    \item $M_{max}^{GarageArea}= 0.2LotArea$
\end{itemize}
\subsection{Baños por cada Dormitorio}
\begin{align}
& 3\,FullBath_i \ge 2\,Bedroom_i
\end{align}

\subsection{Piscina}\\
\subsubsection{El área de la piscina tiene que acotarse al espacio que queda}

\begin{align}
& AreaPool_i \le \Big(LotArea_i - 1stFlrSF_i - GarageArea_i - WoodDeckSF_i - OpenPorchSF_i - EnclosedPorch_i - ScreenPorch_i - 3SsnPorch_i\Big)\cdot HasPool_i,\\
\\
& AreaPool_i \le U_{max}^{Pool}\cdot HasPool_i,\qquad\\
  AreaPool_i \ge U_{min}^{Pool} HasPool_i,\qquad\\
  AreaPool_i \ge 0
\end{align}
\\
Donde:
\begin{itemize}
    \item $U_{min}^{Pool}=160$
    \item $U_{max}^{Pool}= 0.1 LotArea$
\end{itemize}

\subsection{Porch}\\
\subsubsection{Área total del Porch es la suma de todos los Porch}
\begin{align}
& TotalPorchSF_i = OpenPorchSF_i + EnclosedPorch_i + ScreenPorch_i + 3SsnPorch_i,\\
\\
& TotalPorchSF_i \le U_{max}^{TotPorch},\\
\\
& TotalPorchSF_i \le 1stFlrSF_i\\
\end{align}
\\
Donde: 
\begin{itemize}
    \item $U_{max}^{TotPorch} \leq 0.25 LotArea$\\

\subsubsection{Mínimos funcionales por tipo (activados por las binarias que ya declaraste)}\\
\end{itemize}
 
\begin{align}
& OpenPorchSF_i \ge 40 \cdot HasOpenPorch_i,\\
\\
& EnclosedPorch_i \ge 60 \cdot HasEnclosedPorch_i,\\
\\
& ScreenPorch_i \ge 40 \cdot HasScreenPorch_i,\\
\\
& 3SsnPorch_i \ge 80 \cdot Has3SsnPorch_i\\
\end{align}

\subsubsection{Compatibilidad de espacios exteriores}\\
\begin{align}
& WoodDeckSF_i + TotalPorchSF_i + AreaPool_i \le U_{max}^{AreaExt},\\
& WoodDeckSF_i + OpenPorchSF_i \le U_{max}^{WDyPorch}
\end{align}\\
Donde:\\
\begin{itemize}
    \item $U_{max}^{AreaExt}=0.35LotArea$
    \item $U_{max}^{WDyPorch}=0.2LotArea$
\end{itemize}
\subsection{Deck}\\
\begin{align}
& U_{min}^{WD} \cdot HasWoodDeck_i \leq WoodDeckSF_i \leq  U_{max}^{WD}\,LotArea_i \cdot HasWoodDeck_i\\

\end{align}
Donde:\\
\begin{itemize}
    \item $U_{min}^{WD}=40$ valor mínimo de WoodDeck
    \item $U_{max}^{WD}=0.15LotArea$: valor máximo de WoodDeck
\end{itemize}

\subsection{Acabados}
\begin{align}
& TotalBsmtSF_i \;\le\; 0.5\,LotArea_i \,\big(1 - BsmtExposure_{i,NA}\big). \label{eq:bsmt-cap}
\end{align}

 (3) Partición de áreas terminadas (2 canales) --
\begin{align}
& BsmtFinSF1_i + BsmtFinSF2_i \;=\; TotalBsmtSF_i. \label{eq:bsmt-sum}
% Si manejas área sin terminar explícita, usa en su lugar:
% BsmtFinSF1_i + BsmtFinSF2_i + BsmtUnfSF_i = TotalBsmtSF_i.
\end{align}

%--- (4) Activación de áreas terminadas por selección de tipo (sin nuevas binarias)
%     Canal 1: cualquier tipo distinto de NA permite área; "acabado real" = en Bfin
\begin{align}
& BsmtFinSF1_i \;\le\; 0.5\,LotArea_i \!\!\!\sum_{b_1\in \Buno\setminus\{\text{NA}\}}\!\! BsmtFinType1_{i,b_1}, \label{eq:fin1-on}\\
& BsmtFinSF2_i \;\le\; 0.5\,LotArea_i \!\!\!\sum_{b_2\in \Buno\setminus\{\text{NA}\}}\!\! BsmtFinType2_{i,b_2}. \label{eq:fin2-on}
\end{align}

%--- (5) (Opcional) Mínimos funcionales de acabado sólo si el tipo es "acabado" (no Unf ni NA)
\begin{align}
& BsmtFinSF1_i \;\ge\; A^{fin}_{\min} \!\!\sum_{b_1\in \Bfin}\! BsmtFinType1_{i,b_1}, \qquad
  BsmtFinSF2_i \;\ge\; A^{fin}_{\min} \!\!\sum_{b_2\in \Bfin}\! BsmtFinType2_{i,b_2}. \label{eq:fin-min}
\end{align}

%--- (6) Baños en sótano sólo si hay ALGÚN acabado real (en cualquiera de los dos canales)
\begin{align}
& BsmtFullBath_i \;\le\; 2 \left(
     \sum_{b_1\in \Bfin}\! BsmtFinType1_{i,b_1}
   + \sum_{b_2\in \Bfin}\! BsmtFinType2_{i,b_2} \right), \label{eq:bbath-full}\\
& BsmtHalfBath_i \;\le\; 1 \left(
     \sum_{b_1\in \Bfin}\! BsmtFinType1_{i,b_1}
   + \sum_{b_2\in \Bfin}\! BsmtFinType2_{i,b_2} \right). \label{eq:bbath-half}
\end{align}

%--- (7) Apagado total si Exposure = NA (refuerzo, redundante con \eqref{eq:bsmt-cap})
\begin{align}
& BsmtFinSF1_i \;\le\; 0.5\,LotArea_i \,\big(1 - BsmtExposure_{i,NA}\big), \\
& BsmtFinSF2_i \;\le\; 0.5\,LotArea_i \,\big(1 - BsmtExposure_{i,NA}\big), \\
& BsmtFullBath_i \;\le\; 2\,\big(1 - BsmtExposure_{i,NA}\big), \qquad
  BsmtHalfBath_i \;\le\; 1\,\big(1 - BsmtExposure_{i,NA}\big).
\end{align}

% Nota:
%  - \eqref{eq:exp-onehot} y \eqref{eq:exp-exist} reemplazan la lógica con HasBasement_i.
%  - \eqref{eq:bsmt-cap} sustituye TotalBsmtSF_i \le 0.5 LotArea_i * HasBasement_i.
%  - \eqref{eq:fin1-on}–\eqref{eq:fin2-on} sustituyen las restricciones con HasBsmtFin1_i / HasBsmtFin2_i.
%  - \eqref{eq:bbath-full}–\eqref{eq:bbath-half} reemplazan los límites de baños usando sólo la selección de tipos "acabados".


\subsection{Exterior}
\begin{align}
& \sum_{e_1} Exterior1st_{i,e_1} = UseExterior1st_i \qquad (\text{si decides usar }UseExterior1st_i),\\
& \sum_{e_2} Exterior2nd_{i,e_2} = UseExterior2nd_i,\\
& SameMaterial_i \ge Exterior1st_{i,e_1} + Exterior2nd_{i,e_2} - 1 \quad \forall e_1,e_2,\\
& UseExterior2nd_i \le 1 - SameMaterial_i
\end{align}
\subsection{Mamposteria}
\subsubsection{Cota superior Mampostería}
\begin{align}
    MasVnrArea_{i}\leq U_{i}^{mas}
\end{align}\\
Donde:
\begin{itemize}
    \item $U_{i}^{mas}=f_{max}^{mas}\cdot AreaExterior_{i}$
    \item $f_{max}^{mas}=0.4$
\end{itemize}
\begin{align}
    &MasVnrArea_{i}\geq A_{min}^{MasVnr}\cdot (1-MasVnrType_{i,None})
\end{align}\\
Variable auxiliar:\\
\begin{align}
    &MvProd_{i,t}\leq MasVnrArea_{i}\\
    &MvProd_{i,t}\equiv MasVnrArea_{i} \cdot MasVnrType_{i,t}\\
    &MvProd_{i,t} \leq U_{i}^{mas}\cdot MasVnrType_{i,t}\\
    &MvProd_{i,t}\geq MasVnrArea_{i} - U_{i}^{mas}\cdot (1-MasVnrType_{i,t})\\
    &MvProd_{i,t}\geq 0
\end{align}

Donde:\\
\begin{align}

& MasVnrArea_i \le TotalArea_i,\\ \quad MasVnrArea_i \ge 0
\end{align}\\
Donde:\\
\begin{itemize}
    \item $A_{min}^{MasVnr}=20 $ft, Cota inferior Area mampostería
    \item $A_{max}^{MasVnr}=2000$
\end{itemize}


\subsection{Garage}

\begin{align}
& GarageCars_i \le \overline{C}^{\text{cars}} \,\big(1 - GarageType_{i,NA}\big) \qquad \forall i,\\
& GarageArea_i \le \overline{A}^{\text{garage}}_i \,\big(1 - GarageType_{i,NA}\big) \qquad \forall i.
\end{align}
\begin{align}\\
& GarageCars_i \ge 1 - GarageType_{i,NA} \qquad \forall i.
\end{align}\\
\begin{align}
& GarageFinish_{i,NA} = GarageType_{i,NA} \qquad \forall i,\\
& GarageFinish_{i,\text{Fin}} + GarageFinish_{i,\text{RFn}} 
  = 1 - GarageType_{i,NA} \qquad \forall i.
\end{align}
Donde:
\begin{itemize}
    \item $\bar{C}_{i}^{cars}=4$
    \item $\bar{A}_{i}^{garage}=0.2 LotArea$
\end{itemize}
\begin{align}
    &GarageType_{i,NoAplica}=GarageFinish_{i,NoAplica}
\end{align}\\
Ahora es necesario linealizar para calcular los costos:
\\
\begin{align}
    GA_{i,g}\leq GarageArea_{i}\\
    GA_{i,g}\leq \bar A_{i}^{Garage}\cdot GarageType_{i,g}\\
    GA_{i,g}\geq GarageArea_{i}-\bar A_{i}^{Garage}(1-GarageType_{i,g})\\
    GA_{i,g}\geq 0
\end{align}

\subsection{basement}
\textbf{Parámetros:}
\[
U^{bsmt}_i = 0.5\cdot LotArea_i,\qquad
U^{bF}=2,\qquad
U^{bH}=1,\qquad
A^{fin}_{\min}\ge 0 \;.
\]
\subsubsection{Capacidad Máxima del Sótano}
% Variables (ya declaradas en tu documento)
% BsmtFinType1_{i,b1}, BsmtFinType2_{i,b2}, BsmtExposure_{i,x} binarias one–hot
% TotalBsmtSF_i, BsmtFinSF1_i, BsmtFinSF2_i, BsmtUnfSF_i \ge 0
% BsmtFullBath_i, BsmtHalfBath_i enteras \ge 0

% One–hot
\begin{align}
    &BsmtFinSF1_{i}+BsmtFinSF2_{i}=TotalBsmtSF_{i}\\
\end{align}\\
\\
\begin{align}
    &TotalBsmtSF_{i}\leq 0.5 LotArea_{i}(1-BsmtExpoure_{i,NA})\\
\end{align}\\


% Existencia sótano via NA en exposición
\textbf{Existencia de sótano vía exposición \texttt{NA}:}\\
\begin{align}
& TotalBsmtSF_i \;\le\; U^{bsmt}_i \big(1 - BsmtExposure_{i,NA}\big). \label{eq:b-exist}
\end{align}\\
Donde:
\begin{itemize}
    \item $U_{i}^{bsmt}=0.5LotArea$
\end{itemize}
% Partición de áreas


% Activadores no–NA y de acabado real
\textbf{Activadores (definición auxiliar):}
\[
\phi^{(1)}_i \;=\; \sum_{b_1\in B_1\setminus\{\text{NA}\}} BsmtFinType1_{i,b_1}, \qquad
\phi^{(2)}_i \;=\; \sum_{b_2\in B_2\setminus\{\text{NA}\}} BsmtFinType2_{i,b_2},
\]
\[
\psi^{(1)}_i \;=\; \sum_{b_1\in \{\text{GLQ, ALQ, BLQ, Rec, LwQ}\}} BsmtFinType1_{i,b_1}, \qquad \\
\\ \psi^{(2)}_i \;=\; \sum_{b_2\in \{\text{GLQ, ALQ, BLQ, Rec, LwQ}\}} BsmtFinType2_{i,b_2}.
\]
Donde:
\begin{itemize}
    \item $\phi_{i}^{(1)}, \phi_{i}^{(2)}$ indicadores de existencia de cualquier tipo de acabado menos NA
    \item $\psi _{i}^{(1)}, \psi_{i}^{(2)}$: indicadores de existencia de acabado real.
\end{itemize}
% Activación de áreas terminadas por canal
\textbf{Activación y mínimos de acabados por canal:}\\
\begin{align}
& BsmtFinSF1_i \;\le\; U^{bsmt}_i\, \phi^{(1)}_i, \qquad\\
  BsmtFinSF2_i \;\le\; U^{bsmt}_i\, \phi^{(2)}_i, \label{eq:b-on}\\
& BsmtFinSF1_i \;\ge\; A^{fin}_{\min}\, \psi^{(1)}_i, \qquad\\
  BsmtFinSF2_i \;\ge\; A^{fin}_{\min}\, \psi^{(2)}_i. \label{eq:b-min}
\end{align}\\

% Baños condicionados a acabado real
\textbf{Baños en sótano sólo si hay acabado real:}\\
\\
\begin{align}
& BsmtFullBath_i \;\le\; U^{bF}\,\big(\psi^{(1)}_i + \psi^{(2)}_i\big), \qquad \\
  BsmtHalfBath_i \;\le\; U^{bH}\,\big(\psi^{(1)}_i + \psi^{(2)}_i\big). \label{eq:b-baths}
\end{align}

% Apagado completo si exposición NA
\textbf{Apagado completo si \texttt{BsmtExposure\_NA}=1:}\\
\begin{align}
& BsmtFinSF1_i \;\le\; U^{bsmt}_i \big(1 - BsmtExposure_{i,NA}\big), \qquad \\
  BsmtFinSF2_i \;\le\; U^{bsmt}_i \big(1 - BsmtExposure_{i,NA}\big), \\
& BsmtFullBath_i \;\le\; U^{bF}\big(1 - BsmtExposure_{i,NA}\big), \qquad \\
  BsmtHalfBath_i \;\le\; U^{bH}\big(1 - BsmtExposure_{i,NA}\big). \label{eq:b-off}
\end{align}
\begin{itemize}
  \item $U^{\text{bF}}\in\mathbb{Z}_{\ge 0}$: cota superior de baños completos en sótano (p.ej. $2$).
  \item $U^{\text{bH}}\in\mathbb{Z}_{\ge 0}$: cota superior de medios baños en sótano (p.ej. $1$).
  \item $A^{\text{fin}}_{\min}\ge 0$: área mínima funcional para declarar un acabado “real” (p.ej. $100~\text{ft}^2$).
\end{itemize}

%-------------------------------
% (1) Exposición: selección única y existencia vía NA
%-------------------------------
\begin{align}

& \sum_{x\in \{\text{Gd,Av,Mn,No}\}} BsmtExposure_{i,x}
   \;=\; 1 - BsmtExposure_{i,NA}. \label{eq:exp-exist}
\end{align}\\
\\

%-------------------------------
% (2) Capacidad de área total condicionada por NA
%-------------------------------
\begin{align}
& TotalBsmtSF_i \;\le\; U_{i}^{\text{bsmt}}\,\big(1 - BsmtExposure_{i,NA}\big). \label{eq:bsmt-cap}
\end{align}

%-------------------------------
% (3) Partición de áreas terminadas (2 canales)
%     (ajusta si modelas BsmtUnfSF_i explícito)
%-------------------------------
\begin{align}
& BsmtFinSF1_i + BsmtFinSF2_i \;=\; TotalBsmtSF_i. \label{eq:bsmt-sum}
\end{align}

%-------------------------------
% (4) Activación de áreas por selección de tipo (sin nuevas binarias)
%     Cualquier tipo distinto de NA permite área en el canal
%-------------------------------

%-------------------------------
% (5) Mínimos funcionales SOLO para acabados “reales” (no Unf, no NA)
%-------------------------------
\begin{align}
& BsmtFinSF1_i \;\ge\; A^{\text{fin}}_{\min}\; \sum_{b_1\in \Bfin} BsmtFinType1_{i,b_1}, 
\qquad
  BsmtFinSF2_i \;\ge\; A^{\text{fin}}_{\min}\; \sum_{b_2\in \Bfin} BsmtFinType2_{i,b_2}. \label{eq:fin-min}
\end{align}

%-------------------------------
% (6) Baños en sótano sólo si hay ALGÚN acabado real (en alguno de los canales)
%-------------------------------
\begin{align}
& BsmtFullBath_i \;\le\; U^{\text{bF}} \left(
     \sum_{b_1\in \Bfin} BsmtFinType1_{i,b_1}
   + \sum_{b_2\in \Bfin} BsmtFinType2_{i,b_2} \right), \label{eq:bbath-full} \\[6pt]
& BsmtHalfBath_i \;\le\; U^{\text{bH}} \left(
     \sum_{b_1\in \Bfin} BsmtFinType1_{i,b_1}
   + \sum_{b_2\in \Bfin} BsmtFinType2_{i,b_2} \right). \label{eq:bbath-half}
\end{align}

%-------------------------------
% (7) Apagado total si Exposure = NA (refuerzos útiles)
%-------------------------------
\begin{align}
& BsmtFinSF1_i \;\le\; \phi^{\text{bsmt}}\; LotArea_i \,\big(1 - BsmtExposure_{i,NA}\big), \\[6pt]
& BsmtFinSF2_i \;\le\; \phi^{\text{bsmt}}\; LotArea_i \,\big(1 - BsmtExposure_{i,NA}\big), \\[6pt]
& BsmtFullBath_i \;\le\; U^{\text{bF}}\,\big(1 - BsmtExposure_{i,NA}\big), 
\qquad
  BsmtHalfBath_i \;\le\; U^{\text{bH}}\,\big(1 - BsmtExposure_{i,NA}\big).
\end{align}

\subsection{Forzar Material Principal}
\begin{align}
    &Exterior1st_{i}=Exterior2nd_{i}
\end{align}


\subsection{Foundation}
Si la casa es de madera o Slab no puede tener sótano segpun expertos\\
\begin{align}
    Foundation_{i,Slab}\leq BsmtExposure_{i,NA}\\
    Foundation_{i,Wood}\leq BsmtExposure_{i,NA}\\
\end{align}

\begin{align}
    AreaFoundation_{i}=1stFlrSF_{i}\\
    
\end{align}
Hacemos linealización:\\

\\
\begin{align}
    &FA_{i,f}\leq AreaFoundation_{i}\\
    &FA_{i,f}\leq U_{i}^{Found}\cdot Foundation_{i,f}\\
    &FA_{i,f}\geq AreaFoundation_{i} - U_{i}^{Found} \cdot (1-Foundation_{i,f})\\
    &FA_{i,f}\geq 0
\end{align}

Donde:
\begin{itemize}
    \item $U_{i}^{foundation}= 0.6LotArea_{i}$

\end{itemize}


\subsection{Dormitorios}

\subsubsection{Dormitorios 1er piso}
\begin{align}
    &AreaBedroom1_{i} \leq U_{max}^{Bed1} \cdot Floor1_{i}\\
    &AreaFullBath1_{i} \leq U_{max}^{FullB1} \cdot Floor1_{i}\\
    &AreaHalfBath1_{i} \leq U_{max}^{HalfB1} \cdot Floor1_{i}\\
    &AreaKitchen1_{i} \leq U_{max}^{Kitchen1} \cdot Floor1_{i}\\
    &AreaOther1_{i} \leq U_{max}^{Other1} \cdot Floor1_{i}\\
\end{align}

\subsubsection{Dormitorios segundo piso}
\begin{align}
    &AreaBedroom2_{i} \leq U_{max}^{Bed2} \cdot Floor2_{i}\\
    &AreaFullBath2_{i} \leq U_{max}^{FullB2} \cdot Floor2_{i}\\
    &AreaHalfBath2_{i} \leq U_{max}^{HalfB2} \cdot Floor2_{i}\\
    &AreaKitchen2_{i} \leq U_{max}^{Kitchen2} \cdot Floor2_{i}\\
    &AreaOther2_{i} \leq U_{max}^{Other2} \cdot Floor2_{i}\\
\end{align}
\\
Donde:
\begin{itemize}
    \item $U_{max}^{Bed2}=200 ft$
    \item $U_{max}^{FullB2}=60$
    \item $U_{max}^{HalfB2}=20$
    \item $U_{max}^{Kitchen2}=$
    \item $$
\end{itemize}
\\
\begin{align}
    AreaBedroom2_{i}+AreaKitchen2_{i}+AreaHalfBath2_{i}+AreaFullBath2_{i}+AreaOther2_{i}\leq 2ndFlrSF_{i}
\end{align}
\\
\begin{align}
    Remainder2_{i}\geq 0\\
    AreaBedroom2_{i}+AreaKitchen2_{i}+AreaHalfBath2_{i}+AreaFullBath2_{i}+AreaOther2_{i}+Remainder2_{i}= 2ndFlrSF_{i}
\end{align}\\
Donde:
\begin{itemize}
    \item $Remainder2_{i}$ Representa pasillos, closets
\end{itemize}
\\

\subsection{Area Living/Recreación}
\begin{align}
    &OtherRooms_{i}=OtherRooms1_{i}+OtherRooms2_{i}\\
    &AreaOther_{i}=AreaOther1_{i}+AreaOther2_{i}
\end{align}
Donde:\\
\begin{itemize}
    \item $a_{min}^{other}=100$
    \item $R_{max}^{others}=8$
    \item $U_{i}^{others}=2ndFlrSF_{i}$
\end{itemize}
\subsubsection{Incluir estas areas en pisos}
\begin{align}
    TotalRmsAbvGrd_{i}=Bedroom_{i}+FullBath_{i}+HalfBath_{i}+OtherRooms_{i}\\
    OtherRooms_{i}=OtherRooms1_{i}+OtherRooms2_{i}\\
    AreaOther_{i}=AreaOther1_{i}+AreaOther2_{i}
\end{align}
\subsubsection{Presencia mínima en primer piso}
\begin{align}
    OtherRooms1_{i}\geq 1
\end{align}
\subsubsection{Activación en segundo piso}
\begin{align}
    OtherRooms2_{i}\leq R_{max}^{Other}\cdot Floor2_{i}\\
    AreaOther2_{i}\leq U_{i}^{Other2}
\end{align}
\subsubsection{Mínimo de area}
\begin{align}
    AreaOther1_{i}\geq a_{min}^{other}\cdot OtherRooms1_{i}\\
    AreaOther2_{i} \geq a_{min}^{other}\cdot OtherRooms2_{i}
\end{align}


\subsection{Perímetro casa para calcular Exterior}
\begin{align}
    &P_{i}^{(1)}\geq P_{i}^{(2)}
\end{align}
\subsubsection{Area Exterior a curbir}
\begin{align}
    AreaExterior_{i}=H^{ext}\cdot (P_{i}^{(1)}+P_{i}^{(2)})
\end{align}
Donde:
\begin{itemize}
    \item $H^{ext}$=7 ft
    \item $P_{i}^{(1)}\leq 2\cdot \Big ( \frac{1stFlrSF_{i}}{s_{min}}+s_{min}\Big)\cdot Floor1_{i}$
    \item $P_{i}^{(2)}\leq 2\cdot \Big ( \frac{2ndFlrSF_{i}}{s_{min}}+s_{min}\Big)\cdot Floor2_{i}$
    \item $P_{i}^{(1)}\geq 2\cdot \Big ( \frac{1stFlrSF_{i}}{s_{max}}+s_{max}\Big)\cdot Floor1_{i}$
    \item $P_{i}^{(2)}\geq2\cdot \Big ( \frac{2ndFlrSF_{i}}{s_{max}}+s_{max}\Big)\cdot Floor2_{i}$
    \item $s_{min}=20$
    \item $s_{max}=70$
\end{itemize}
Este tipo de relación es estándar en la literatura de geometría arquitectónica y modelación espacial (Shpuza, 2006; Smith, 2017; March, 1972; Gero & Rosenman, 1985), donde el perímetro se expresa como una función del área y la proporción mínima de lados para mantener formas constructivamente viables.\\
\\
Hago nueva variable W para linealizar:\\
\begin{align}
    &W_{i,e1}\leq AreaExterior_{i}\\
    &W_{i,e1}\leq U_{i}^{ext}\cdot Exterior1st_{i,e1}\\
    &W_{i,e1}\geq AreaExterior_{i}-U_{i}^{ext}(1-Exterior1st_{i,e1})\\
    &\sum_{e1}W_{i,e1}=AreaExterior_{i}
\end{align}\\
Donde:\\
\begin{itemize}
    \item $U_{i}^{ext}=4500$
\end{itemize}




\subsection*{Parámetros de superficie real del techo}

El parámetro $\gamma_{s,m}$ ajusta el área en planta cubierta por el techo
($PlanRoofArea_i$) para reflejar el área real de material requerido, considerando
la pendiente, el solape y la geometría asociada al estilo y material del techo.
Los valores se basan en estándares de la \textit{National Association of Home Builders} (NAHB, 2023),
la \textit{Asphalt Roofing Manufacturers Association} (ARMA, 2021) y el
\textit{Roofing Alliance Technical Guide} (2022).

\begin{table}[H]
\centering
\caption{Factores de superficie real del techo ($\gamma_{s,m}$) según estilo y material.}
\label{tab:roof_gamma}
\begin{tabular}{lllcc}
\toprule
\textbf{Estilo ($s$)} & \textbf{Material ($m$)} & \textbf{Descripción} & $\boldsymbol{\gamma_{s,m}}$ & \textbf{Fuente}\\
\midrule
Flat        & Membran      & Techo plano o de losa con mínima pendiente & 1.00 & NAHB (2023)\\
Flat        & CompShg      & Plano con tejas asfálticas & 1.05 & ARMA (2021)\\
Gable       & CompShg      & A dos aguas estándar (4:12–6:12 pitch) & 1.10 & NAHB (2023)\\
Gable       & Metal        & A dos aguas con panel metálico & 1.12 & Roofing Alliance (2022)\\
Hip         & CompShg      & A cuatro aguas (moderada pendiente) & 1.15 & NAHB (2023)\\
Hip         & Metal        & A cuatro aguas con panel metálico & 1.18 & Roofing Alliance (2022)\\
Gambrel     & WdShake      & Tipo granero, tejas de madera & 1.25 & NAHB (2023)\\
Mansard     & CompShg      & Mansarda con inclinación alta & 1.28 & NAHB (2023)\\
Shed        & Metal        & Techo inclinado de una sola vertiente & 1.12 & ARMA (2021)\\
Gable       & ClyTile      & A dos aguas con tejas de arcilla & 1.20 & Roofing Alliance (2022)\\
Hip         & TarGrv       & A cuatro aguas con grava asfáltica & 1.10 & NAHB (2023)\\
\bottomrule
\end{tabular}
\end{table}

Los valores típicos oscilan entre $1.00$ (techo plano) y $1.30$ (techo muy inclinado
o con múltiples vertientes). En la práctica, $\gamma_{s,m}$ puede estimarse como:
\[
\gamma_{s,m} \approx 1 + 0.1 \cdot \tan(\theta_s)
\]
donde $\theta_s$ es el ángulo medio de pendiente del estilo de techo.

\paragraph{Fuentes bibliográficas:}
\begin{itemize}
    \item National Association of Home Builders (NAHB). (2023). \textit{Residential Construction Guidelines, 2023 Edition.} Washington, D.C.
    \item Asphalt Roofing Manufacturers Association (ARMA). (2021). \textit{Residential Asphalt Roofing Manual, 2021 Edition.}
    \item Roofing Alliance. (2022). \textit{Technical Guide to Roof System Performance and Design.} National Roofing Contractors Association.
\end{itemize}

\paragraph{Conjuntos} 
$S$: estilos de techo;\quad $M$: materiales de techo.

\paragraph{Parámetros (dados)}
\[
U^{(1)}_i = \bar{\alpha}_{1}\, LotArea_i,\qquad
U^{(2)}_i = \bar{\alpha}_{2}\, LotArea_i,\qquad
U^{\text{plan}}_i = \bar{\alpha}_{\text{plan}}\, LotArea_i,
\]
\[
\gamma_{s,m}\ge 1\quad \forall s\in S,\, m\in M,
\]
donde típicamente $\bar{\alpha}_{1}=0.6,\ \bar{\alpha}_{2}=0.5,\ \bar{\alpha}_{\text{plan}}=0.6$ (ajustables). 
% (Opcional) Matriz de compatibilidad estilo–material: A_{s,m}\in\{0,1\}.

\paragraph{Variables}
\[
Floor1_i,\, Floor2_i \in \{0,1\};\quad
PR1_i,\, PR2_i \ge 0;\quad
PlanRoofArea_i \ge 0;\quad
ActualRoofArea_i \ge 0;
\]
\[
RoofStyle_{i,s}\in\{0,1\}\ \forall s\in S;\quad
RoofMatl_{i,m}\in\{0,1\}\ \forall m\in M;
\]
\[
Y_{i,s,m}\in\{0,1\},\ Z_{i,s,m}\ge 0\quad \forall s\in S,\ \forall m\in M.
\]

%==============================
% Restricciones
%==============================

% (1) Elección de número de pisos
\[
Floor1_i + Floor2_i = 1. 
\]\\[6pt]

% (2) Área en planta a cubrir por el techo (linealización)
\[
PR1_i \le 1stFlrSF_i,\qquad 
PR1_i \le U^{(1)}_i\, Floor1_i,
\]
\[
PR1_i \ge 1stFlrSF_i - U^{(1)}_i(1-Floor1_i),\qquad 
PR1_i \ge 0,
\]\\[6pt]
\[
PR2_i \le 2ndFlrSF_i,\qquad 
PR2_i \le U^{(2)}_i\, Floor2_i,
\]
\[
PR2_i \ge 2ndFlrSF_i - U^{(2)}_i(1-Floor2_i),\qquad 
PR2_i \ge 0,
\]\\[6pt]
\[
PlanRoofArea_i = PR1_i + PR2_i,\qquad 
0 \le PlanRoofArea_i \le U^{\text{plan}}_i.
\]\\[10pt]

% (3) Selección única de estilo y material
\[
\sum_{s\in S} RoofStyle_{i,s} = 1,\qquad 
\sum_{m\in M} RoofMatl_{i,m} = 1.
\]\\[6pt]

% (Opcional) Compatibilidad estilo–material
% \[
% Y_{i,s,m} \le A_{s,m}\quad \forall s,m.
% \]\\[6pt]

% (4) Conjunción estilo–material (AND lógico)
\[
Y_{i,s,m} \le RoofStyle_{i,s},\qquad 
Y_{i,s,m} \le RoofMatl_{i,m}\quad \forall s,m,
\]
\[
Y_{i,s,m} \ge RoofStyle_{i,s} + RoofMatl_{i,m} - 1\quad \forall s,m,
\]
\[
\sum_{s\in S}\sum_{m\in M} Y_{i,s,m} = 1.
\]\\[10pt]

% (5) Linealización de Z_{i,s,m} \approx PlanRoofArea_i \cdot Y_{i,s,m}
\[
Z_{i,s,m} \le PlanRoofArea_i,\qquad 
Z_{i,s,m} \le U^{\text{plan}}_i\, Y_{i,s,m}\quad \forall s,m,
\]
\[
Z_{i,s,m} \ge PlanRoofArea_i - U^{\text{plan}}_i(1 - Y_{i,s,m})\quad \forall s,m,
\]
\[
Z_{i,s,m} \ge 0\quad \forall s,m.
\]\\[10pt]

% (6) Área real de techo (igualdad principal)
\[
ActualRoofArea_i \;=\; \sum_{s\in S}\sum_{m\in M} \gamma_{s,m}\, Z_{i,s,m}.
\]\\[8pt]

% (Opcional) Fortalecimientos por acotación (si conoces \gamma_{\max}=\max_{s,m}\gamma_{s,m})
% \[
% ActualRoofArea_i \ge PlanRoofArea_i,\qquad 
% ActualRoofArea_i \le \gamma_{\max}\, PlanRoofArea_i.
% \]\\[6pt]

% (Opcional) Costo del techo
% \[
% C^{techo}_i = c^{roof}\cdot ActualRoofArea_i.
% \]



\\
\\
\\
\section{Variables Binarias}
\begin{itemize}
    \item $Floor1_{i},\, Floor2_{i}$ (piso único / dos pisos)
    \item $HasPool_{i}$, $HasWoodDeck_{i}$
    \item $HasOpenPorch_{i}$, $HasEnclosedPorch_{i}$, $Has3SsnPorch_{i}$, $HasScreenPorch_{i}$
    \item $HasFireplaces_{i}$, $HasPavedDrive_{i}$, $HasReja_{i}$
    \item $SameMaterial_{i}$ (revestimientos exterior1/exterior2 iguales)
    % Estilo y material de techo; sistemas; etc. se definen como one–hot en las restricciones
\end{itemize}

\section{Variables de Área}
\begin{itemize}
    \item $AreaBedrooms_{i} \in \mathbb{Q}_{\ge 0}$
    \item $AreaKitchen_{i},\; AreaFullBath_{i},\; AreaHalfBath_{i}$
    \item $AreaKitchen1_{i},\; AreaKitchen2_{i}$
    \item $AreaFullBath1_{i},\; AreaFullBath2_{i}$
    \item $AreaHalfBath1_{i},\; AreaHalfBath2_{i}$
    \item $AreaFoundation_{i}$
    \item $AreaRoof_{i}$ \; (área en planta a cubrir por el techo)
    \item $PR1_i,\, PR2_i \ge 0$ \quad (\textit{auxiliares para linealizar $AreaRoof_i$})
    \item $OpenPorchSF_{i},\; EnclosedPorch_{i},\; ScreenPorch_{i},\; 3SsnPorch_{i},\; WoodDeckSF_{i}$
    \item $TotalPorchSF_{i},\; AreaPool_{i}$
\end{itemize}

\section{Variables de Conteo}
\begin{itemize}
    \item $FullBath1_{i},\, FullBath2_{i}$ \quad (baños completos por piso)
    \item $HalfBath1_{i},\, HalfBath2_{i}$ \quad (medios baños por piso)
    \item $Kitchen1_{i},\, Kitchen2_{i}$ \quad (cocinas por piso)
\end{itemize}

% ============================
% EXCLUSIVIDADES (ONE-HOT)
% ============================
\section{Restricciones}
\subsection{Exclusividades (one–hot)}
\begin{align}
  &\sum_{s} MSSubClass_{i,s}=1, \quad
  \sum_{u} Utilities_{i,u}=1, \quad
  \sum_{b} BldgType_{i,b}= 1, \quad
  \sum_{hs} HouseStyle_{i,hs}=1, \\[4pt]
  &\sum_{r} RoofStyle_{i,r}=1, \quad
  \sum_{m} RoofMatl_{i,m}=1, \quad
  \sum_{e1} Exterior1st_{i,e1}=1, \quad
  \sum_{e2} Exterior2nd_{i,e2}=1, \\[4pt]
  &\sum_{t} MasVnrType_{i,t}=1, \quad
  \sum_{f} Foundation_{i,f}=1, \quad
  \sum_{x} BsmtExposure_{i,x}=1, \\[4pt]
  &\sum_{b1} BsmtFinType1_{i,b1}=1, \quad
  \sum_{b2} BsmtFinType2_{i,b2}=1, \\[4pt]
  &\sum_{h} Heating_{i,h}= 1,\quad
  \sum_{a} CentralAir_{i,a}=1, \quad
  \sum_{e} Electrical_{i,e}=1, \\[4pt]
  &\sum_{g} GarageType_{i,g}=1,\quad
  \sum_{gf} GarageFinish_{i,gf}=1,\quad
  \sum_{p} PavedDrive_{i,p}=1,\quad
  \sum_{misc} MiscFeature_{i,misc}=1.
\end{align}

% ============================
% CONSISTENCIA DE ÁREAS
% ============================
\subsection{Consistencia de Áreas}
\subsubsection{Construido vs. Lote}
\begin{align}
1stFlrSF_{i} + TotalPorchSF_{i} + AreaPool_{i} \leq LotArea_{i} \qquad \forall i.
\end{align}

\subsubsection{Segundo piso no mayor al primero}
\begin{align}
2ndFlrSF_{i} \leq 1stFlrSF_{i} \qquad \forall i.
\end{align}

\subsubsection{Área habitable}
\begin{align}
GrLivArea_{i} = 1stFlrSF_{i} + 2ndFlrSF_{i} \qquad \forall i.
\end{align}

% ============================
% CONSISTENCIAS DE ÁREAS POR AMBIENTE
% ============================
\subsection{Consistencias por Ambiente}
\begin{align}
AreaFullBath_{i} &= AreaFullBath1_{i}+ AreaFullBath2_{i}, \\[2pt]
AreaHalfBath_{i} &= AreaHalfBath1_{i}+ AreaHalfBath2_{i}.
\end{align}

% ============================
% FUNCIONALIDAD MÍNIMA
% ============================
\subsection{Funcionalidad mínima}
\begin{align}
FullBath1_{i}\geq 1, \qquad Kitchen1_{i}\geq 1 \qquad \forall i.
\end{align}

% ============================
% CONSISTENCIA DE CANTIDADES
% ============================
\subsection{Consistencias de Cantidades}
\begin{align}
FullBath_i &= FullBath1_i + FullBath2_i, \\[2pt]
HalfBath_i &= HalfBath1_i + HalfBath2_i, \\[2pt]
Kitchen_i  &= Kitchen1_i  + Kitchen2_i \qquad \forall i.
\end{align}

% ============================
% MÁXIMOS POR TIPO DE VIVIENDA
% ============================
\subsection{Máximo de Repeticiones por Tipo}
\paragraph{Parámetros}
\begin{align*}
&B_{\max}^{1Fam}=6,\; B_{\max}^{TwnhsE}=4,\; B_{\max}^{TwnhsI}=4,\; B_{\max}^{Dplx}=5,\; B_{\max}^{2FmCon}=8,\\
&F_{\max}^{1Fam}=4,\; F_{\max}^{TwnhsE}=3,\; F_{\max}^{TwnhsI}=3,\; F_{\max}^{Dplx}=4,\; F_{\max}^{2FmCon}=6,\\
&H_{\max}^{1Fam}=2,\; H_{\max}^{TwnhsE}=2,\; H_{\max}^{TwnhsI}=2,\; H_{\max}^{Dplx}=2,\; H_{\max}^{2FmCon}=3,\\
&K_{\max}^{1Fam}=1,\; K_{\max}^{TwnhsE}=1,\; K_{\max}^{TwnhsI}=1,\; K_{\max}^{Dplx}=2,\; K_{\max}^{2FmCon}=2,\\
&Ch_{\max}^{1Fam}=1,\; Ch_{\max}^{TwnhsE}=1,\; Ch_{\max}^{TwnhsI}=1,\; Ch_{\max}^{Dplx}=1,\; Ch_{\max}^{2FmCon}=2.
\end{align*}
\paragraph{Restricciones}
\begin{align}
Bedrooms_{i}&\leq \sum_{b} B_{\max}^{b}\, BldgType_{i,b},\\
FullBath_{i}&\leq \sum_{b} F_{\max}^{b}\, BldgType_{i,b},\\
HalfBath_{i}&\leq \sum_{b} H_{\max}^{b}\, BldgType_{i,b},\\
Kitchen_{i}&\leq \sum_{b} K_{\max}^{b}\, BldgType_{i,b},\\
FirePlaces_{i}&\leq \sum_{b} Ch_{\max}^{b}\, BldgType_{i,b}.
\end{align}

% ============================
% GARAGE
% ============================
\subsection{Garage}
\paragraph{Consistencia área–capacidad}
\begin{align}
150 \cdot GarageCars_{i} \;\le\; GarageArea_{i} \;\le\; 250 \cdot GarageCars_{i} \qquad \forall i.
\end{align}
\paragraph{Sin binaria HasGarage (vía tipo NA)}
\begin{align}
GarageCars_i &\le \overline{C}^{\text{cars}} \,\big(1 - GarageType_{i,NA}\big),\\
GarageArea_i &\le \overline{A}^{\text{garage}}_i \,\big(1 - GarageType_{i,NA}\big),\\
GarageCars_i &\ge 1 - GarageType_{i,NA},\\
GarageFinish_{i,NA} &= GarageType_{i,NA},\\
GarageFinish_{i,\text{Fin}} + GarageFinish_{i,\text{RFn}} + GarageFinish_{i,\text{Unf}}
  &= 1 - GarageType_{i,NA}.
\end{align}
\textbf{Parámetros:}\;
$\overline{C}^{\text{cars}}=4,$\quad $\overline{A}^{\text{garage}}_i=0.2\,LotArea_i$.

% ============================
% CERCA / REJA
% ============================
\subsection{Cerca / Reja}
\paragraph{Parámetro} $C^{\text{Fence}}$ (costo por pie lineal).
\begin{align}
C^{\text{Reja}}_{i} \;=\; C^{\text{Fence}}\cdot LotFrontage_{i}\cdot HasReja_{i}.
\end{align}

% ============================
% TECHO
% ============================
\subsection{Techo: definición de áreas y dependencia de estilo/material}

\paragraph{Conjuntos} 
$S$: estilos de techo;\quad $M$: materiales de techo.

\paragraph{Parámetros (dados)}
\begin{align*}
&U^{(1)}_i = \bar{\alpha}_{1}\, LotArea_i,\qquad
U^{(2)}_i = \bar{\alpha}_{2}\, LotArea_i,\qquad
U^{\text{plan}}_i = \bar{\alpha}_{\text{plan}}\, LotArea_i,\\
&\bar{\alpha}_{1}=0.6,\;\bar{\alpha}_{2}=0.5,\;\bar{\alpha}_{\text{plan}}=0.6\; (\text{ajustables}),\\
&\gamma_{s,m}\ge 1\quad \forall s\in S,\, m\in M.
\end{align*}

\paragraph{Variables}
\begin{align*}
&Floor1_i,\, Floor2_i \in \{0,1\},\quad
PR1_i,\, PR2_i \ge 0,\quad
PlanRoofArea_i \ge 0,\quad
ActualRoofArea_i \ge 0,\\
&RoofStyle_{i,s}\in\{0,1\}\ \forall s\in S,\quad
RoofMatl_{i,m}\in\{0,1\}\ \forall m\in M,\\
&Y_{i,s,m}\in\{0,1\},\quad Z_{i,s,m}\ge 0\quad \forall s,m.
\end{align*}

\paragraph{Elección de número de pisos}
\begin{align}
Floor1_i + Floor2_i = 1.
\end{align}

\paragraph{Área en planta a cubrir por el techo (linealización)}
\begin{align}
PR1_i &\le 1stFlrSF_i, &
PR1_i &\le U^{(1)}_i\, Floor1_i,\\
PR1_i &\ge 1stFlrSF_i - U^{(1)}_i(1-Floor1_i), &
PR1_i &\ge 0, \\[6pt]
PR2_i &\le 2ndFlrSF_i, &
PR2_i &\le U^{(2)}_i\, Floor2_i,\\
PR2_i &\ge 2ndFlrSF_i - U^{(2)}_i(1-Floor2_i), &
PR2_i &\ge 0, \\[6pt]
PlanRoofArea_i &= PR1_i + PR2_i, \qquad 0 \le PlanRoofArea_i \le U^{\text{plan}}_i.
\end{align}

\paragraph{Alias}
\begin{align}
AreaRoof_i \;\equiv\; PlanRoofArea_i.
\end{align}

\paragraph{Selección única de estilo y material}
\begin{align}
\sum_{s\in S} RoofStyle_{i,s} = 1, \qquad 
\sum_{m\in M} RoofMatl_{i,m} = 1.
\end{align}

\paragraph{Conjunción estilo–material (AND lógico)}
\begin{align}
Y_{i,s,m} &\le RoofStyle_{i,s}, &
Y_{i,s,m} &\le RoofMatl_{i,m} \qquad \forall s,m,\\
Y_{i,s,m} &\ge RoofStyle_{i,s} + RoofMatl_{i,m} - 1 \qquad \forall s,m,\\
\sum_{s\in S}\sum_{m\in M} Y_{i,s,m} &= 1.
\end{align}

\paragraph{Linealización de $Z_{i,s,m}\approx PlanRoofArea_i\cdot Y_{i,s,m}$}
\begin{align}
Z_{i,s,m} &\le PlanRoofArea_i, &
Z_{i,s,m} &\le U^{\text{plan}}_i\, Y_{i,s,m} \qquad \forall s,m,\\
Z_{i,s,m} &\ge PlanRoofArea_i - U^{\text{plan}}_i(1 - Y_{i,s,m}) \qquad \forall s,m,\\
Z_{i,s,m} &\ge 0 \qquad \forall s,m.
\end{align}

\paragraph{Área real de techo (igualdad principal)}
\begin{align}
ActualRoofArea_i \;=\; \sum_{s\in S}\sum_{m\in M} \gamma_{s,m}\, Z_{i,s,m}.
\end{align}

\paragraph{(Opcional) Costo de techo}
\begin{align}
C^{\text{techo}}_i = c^{\text{roof}} \cdot ActualRoofArea_i.
\end{align}

\subsection*{Parámetros de superficie real del techo}

El parámetro $\gamma_{s,m}$ ajusta $PlanRoofArea_i$ para reflejar el área real de material requerido, considerando
pendiente, solapes y geometría (estilo y material).
\begin{table}[H]
\centering
\caption{Factores de superficie real del techo ($\gamma_{s,m}$) según estilo y material.}
\label{tab:roof_gamma}
\begin{tabular}{lllcc}
\toprule
\textbf{Estilo ($s$)} & \textbf{Material ($m$)} & \textbf{Descripción} & $\boldsymbol{\gamma_{s,m}}$ & \textbf{Fuente}\\
\midrule
Flat        & Membran      & Techo plano o de losa con mínima pendiente & 1.00 & NAHB (2023)\\
Flat        & CompShg      & Plano con tejas asfálticas & 1.05 & ARMA (2021)\\
Gable       & CompShg      & A dos aguas estándar (4:12–6:12) & 1.10 & NAHB (2023)\\
Gable       & Metal        & A dos aguas con panel metálico & 1.12 & Roofing Alliance (2022)\\
Hip         & CompShg      & A cuatro aguas (moderada pendiente) & 1.15 & NAHB (2023)\\
Hip         & Metal        & A cuatro aguas con panel metálico & 1.18 & Roofing Alliance (2022)\\
Gambrel     & WdShake      & Tipo granero, tejas de madera & 1.25 & NAHB (2023)\\
Mansard     & CompShg      & Mansarda con inclinación alta & 1.28 & NAHB (2023)\\
Shed        & Metal        & Una sola vertiente & 1.12 & ARMA (2021)\\
Gable       & ClyTile      & A dos aguas con teja de arcilla & 1.20 & Roofing Alliance (2022)\\
Hip         & TarGrv       & A cuatro aguas con grava asfáltica & 1.10 & NAHB (2023)\\
\bottomrule
\end{tabular}
\end{table}

Valores típicos: $1.00$ (plano) a $1.30$ (inclinación alta/múltiples vertientes). Una estimación simple:
\[
\gamma_{s,m} \approx 1 + 0.1 \cdot \tan(\theta_s),
\]
donde $\theta_s$ es el ángulo medio de pendiente del estilo de techo.

\paragraph{Fuentes bibliográficas:}
\begin{itemize}
    \item National Association of Home Builders (NAHB). (2023). \textit{Residential Construction Guidelines, 2023 Edition.}
    \item Asphalt Roofing Manufacturers Association (ARMA). (2021). \textit{Residential Asphalt Roofing Manual, 2021 Edition.}
    \item Roofing Alliance. (2022). \textit{Technical Guide to Roof System Performance and Design.}
\end{itemize}

% ============================
% LÍMITES GLOBALES
% ============================
\subsection{Consistencias y Límites Globales}
\begin{align}
TotalBsmtSF_i &= BsmtFinSF1_i + BsmtFinSF2_i,\\
TotalArea_i   &= 1stFlrSF_i + 2ndFlrSF_i + TotalBsmtSF_i,
\end{align}
\begin{align}
1stFlrSF_i &\le 0.6\,LotArea_i, &
2ndFlrSF_i &\le 0.5\,LotArea_i,\\
TotalBsmtSF_i &\le 0.5\,LotArea_i, &
GrLivArea_i &\le 0.8\,LotArea_i,\\
GarageArea_i &\le 0.2\,LotArea_i. &&
\end{align}

% ============================
% REGLAS DE BAÑOS
% ============================
\subsection{Relaciones de baños}
\begin{align}
HalfBath_i \le FullBath_i, \qquad 3\,FullBath_i \ge 2\,Bedroom_i \qquad \forall i.
\end{align}

% ============================
% PISCINA
% ============================
\subsection{Piscina}
\begin{align}
AreaPool_i &\le \Big(LotArea_i - 1stFlrSF_i - GarageArea_i - WoodDeckSF_i - OpenPorchSF_i - EnclosedPorch_i - ScreenPorch_i - 3SsnPorch_i\Big)\cdot HasPool_i,\\
AreaPool_i &\le 0.1\,LotArea_i \cdot HasPool_i,\\
AreaPool_i &\ge 160 \cdot HasPool_i,\\
AreaPool_i &\ge 0.
\end{align}

% ============================
% PORCHES Y DECK
% ============================
\subsection{Porches y Deck}
\begin{align}
TotalPorchSF_i &= OpenPorchSF_i + EnclosedPorch_i + ScreenPorch_i + 3SsnPorch_i,\\
TotalPorchSF_i &\le 0.25\,LotArea_i,\\
TotalPorchSF_i &\le 1stFlrSF_i,
\end{align}
\begin{align}
OpenPorchSF_i &\ge 40 \cdot HasOpenPorch_i,\\
EnclosedPorch_i &\ge 60 \cdot HasEnclosedPorch_i,\\
ScreenPorch_i &\ge 40 \cdot HasScreenPorch_i,\\
3SsnPorch_i &\ge 80 \cdot Has3SsnPorch_i,
\end{align}
\begin{align}
WoodDeckSF_i + TotalPorchSF_i + AreaPool_i &\le 0.35\,LotArea_i,\\
WoodDeckSF_i + OpenPorchSF_i &\le 0.20\,LotArea_i,
\end{align}
\begin{align}
WoodDeckSF_i &\ge 40 \cdot HasWoodDeck_i,\\
WoodDeckSF_i &\le 0.15\,LotArea_i \cdot HasWoodDeck_i.
\end{align}

% ============================
% SÓTANO SIN HasBasement
% ============================
\subsection{Sótano (sin binaria HasBasement)}
\paragraph{Conjuntos}
$B_1,B_2=\{\text{GLQ, ALQ, BLQ, Rec, LwQ, Unf, NA}\}$;\;
$X=\{\text{Gd, Av, Mn, No, NA}\}$;\;
$B_{\mathrm{fin}}=\{\text{GLQ, ALQ, BLQ, Rec, LwQ}\}$.

\paragraph{Parámetros}
\[
U^{bsmt}_i = 0.5\cdot LotArea_i,\qquad
U^{bF}=2,\qquad
U^{bH}=1,\qquad
A^{fin}_{\min}\ge 0 \;(\text{p.ej. }100\text{ ft}^2).
\]

\paragraph{Selección única y existencia vía NA}
\begin{align}
\sum_{b_1\in B_1} BsmtFinType1_{i,b_1} &= 1, &
\sum_{b_2\in B_2} BsmtFinType2_{i,b_2} &= 1, &
\sum_{x\in X} BsmtExposure_{i,x} &= 1,\\
\sum_{x\in \{\text{Gd,Av,Mn,No}\}} BsmtExposure_{i,x}
   &= 1 - BsmtExposure_{i,NA}. &&
\end{align}

\paragraph{Capacidad de área total condicionada por NA}
\begin{align}
TotalBsmtSF_i \;\le\; U^{bsmt}_i \big(1 - BsmtExposure_{i,NA}\big).
\end{align}

\paragraph{Partición de áreas}
\begin{align}
BsmtFinSF1_i + BsmtFinSF2_i = TotalBsmtSF_i.
\end{align}

\paragraph{Activación por selección de tipo (sin nuevas binarias)}
\begin{align}
BsmtFinSF1_i &\le U^{bsmt}_i \!\!\!\sum_{b_1\in B_1\setminus\{\text{NA}\}}\!\! BsmtFinType1_{i,b_1},\\
BsmtFinSF2_i &\le U^{bsmt}_i \!\!\!\sum_{b_2\in B_2\setminus\{\text{NA}\}}\!\! BsmtFinType2_{i,b_2}.
\end{align}

\paragraph{Mínimos funcionales sólo si hay acabado real}
\begin{align}
BsmtFinSF1_i &\ge A^{fin}_{\min} \sum_{b_1\in B_{\mathrm{fin}}} BsmtFinType1_{i,b_1},\\
BsmtFinSF2_i &\ge A^{fin}_{\min} \sum_{b_2\in B_{\mathrm{fin}}} BsmtFinType2_{i,b_2}.
\end{align}

\paragraph{Baños en sótano sólo si hay acabado real}
\begin{align}
BsmtFullBath_i &\le U^{bF}\!\left(
     \sum_{b_1\in B_{\mathrm{fin}}} BsmtFinType1_{i,b_1}
   + \sum_{b_2\in B_{\mathrm{fin}}} BsmtFinType2_{i,b_2} \right),\\
BsmtHalfBath_i &\le U^{bH}\!\left(
     \sum_{b_1\in B_{\mathrm{fin}}} BsmtFinType1_{i,b_1}
   + \sum_{b_2\in B_{\mathrm{fin}}} BsmtFinType2_{i,b_2} \right).
\end{align}

\paragraph{Apagado completo si Exposure=NA (refuerzo)}
\begin{align}
BsmtFinSF1_i &\le U^{bsmt}_i \big(1 - BsmtExposure_{i,NA}\big),\\
BsmtFinSF2_i &\le U^{bsmt}_i \big(1 - BsmtExposure_{i,NA}\big),\\
BsmtFullBath_i &\le U^{bF}\big(1 - BsmtExposure_{i,NA}\big),\qquad
BsmtHalfBath_i \le U^{bH}\big(1 - BsmtExposure_{i,NA}\big).
\end{align}

\section*{Resumen de Parámetros del Modelo de Construcción}

\begin{table}[H]
\centering
\caption{Definición y valores de referencia de los parámetros del modelo.}
\begin{tabular}{lllcl}
\toprule
\textbf{Símbolo} & \textbf{Descripción} & \textbf{Unidades} & \textbf{Valor típico} & \textbf{Fuente / Observación} \\
\midrule
$A_{\min}^{\text{fin}}$ & Área mínima para considerar un acabado real en el sótano &
ft$^2$ & 100 &
NAHB (2023), HUD (2020) \\

$U^{bF}$ & Cota superior de baños completos en sótano &
conteo & 2 &
Código IRC Sección R305, práctica común\\

$U^{bH}$ & Cota superior de medios baños en sótano &
conteo & 1 &
Práctica residencial estándar \\

$U^{\text{bsmt}}_i$ & Cota superior del área total de sótano ($=\phi^{\text{bsmt}}\cdot LotArea_i$) &
ft$^2$ & $0.5\cdot LotArea_i$ &
IRC (2021), NAHB (2023) \\

$\phi^{\text{bsmt}}$ & Proporción máxima del lote que puede ocupar el sótano &
-- & 0.5 &
Criterio típico de edificaciones unifamiliares \\

$U^{(1)}_i$ & Cota superior para área del primer piso ($=\bar\alpha_1\cdot LotArea_i$) &
ft$^2$ & $0.6\cdot LotArea_i$ &
HUD (2020), NAHB (2023) \\

$U^{(2)}_i$ & Cota superior para área del segundo piso ($=\bar\alpha_2\cdot LotArea_i$) &
ft$^2$ & $0.5\cdot LotArea_i$ &
HUD (2020), NAHB (2023) \\

$U^{\text{plan}}_i$ & Cota superior del área en planta cubierta por el techo &
ft$^2$ & $0.6\cdot LotArea_i$ &
Proporción coherente con primer piso \\

$\bar\alpha_1,\bar\alpha_2,\bar\alpha_{\text{plan}}$ & Coeficientes de área máxima relativa por piso &
-- & 0.6,\ 0.5,\ 0.6 &
Ajustables según normativa o base empírica \\

$\gamma_{s,m}$ & Factor de superficie real del techo según estilo $s$ y material $m$ &
-- & 1.00–1.30 &
NAHB (2023), ARMA (2021), Roofing Alliance (2022) \\

$c^{\text{roof}}$ & Costo unitario del techo por ft$^2$ de superficie real &
USD/ft$^2$ & 5–15 &
Estimado según material y mano de obra \\

$C^{\text{Fence}}$ & Costo unitario de instalación de cerca perimetral &
USD/ft lineal & 20–40 &
NAHB Remodeling Guidelines (2023) \\

$\overline{C}^{\text{cars}}$ & Capacidad máxima de autos por garaje &
autos & 4 &
Límite típico en Ames Housing \\

$\overline{A}^{\text{garage}}_i$ & Área máxima de garaje ($0.2\cdot LotArea_i$) &
ft$^2$ & $0.2\cdot LotArea_i$ &
Límite empírico (De Cock, 2011) \\
\bottomrule
\end{tabular}
\end{table}

\paragraph{Notas:}
\begin{itemize}
    \item Los valores pueden calibrarse según zonificación, estándares de diseño o distribución empírica de la base \textit{Ames Housing}.
    \item Los parámetros $\bar\alpha$ y $\phi^{\text{bsmt}}$ definen proporciones máximas de ocupación del terreno por nivel o sótano, mientras que $A_{\min}^{\text{fin}}$ y $U^{bF}$–$U^{bH}$ controlan viabilidad y consistencia funcional interna.
    \item Los factores $\gamma_{s,m}$ permiten convertir el área en planta del techo ($PlanRoofArea_i$) en el área real de material considerando inclinación y tipo constructivo.
\end{itemize}

\paragraph{Fuentes:}
\begin{itemize}
    \item International Code Council (2021). \textit{International Residential Code for One- and Two-Family Dwellings (IRC 2021)}.
    \item U.S. Department of Housing and Urban Development (2020). \textit{Minimum Property Standards for One- and Two-Family Dwellings (HUD Handbook 4910.1)}.
    \item National Association of Home Builders (NAHB). (2023). \textit{Residential Construction Guidelines, 2023 Edition}.
    \item Asphalt Roofing Manufacturers Association (ARMA). (2021). \textit{Residential Asphalt Roofing Manual}.
    \item Roofing Alliance. (2022). \textit{Technical Guide to Roof System Performance and Design}.
    \item De Cock, D. (2011). \textit{Ames, Iowa: Alternative to the Boston Housing Data}. Iowa State University.
\end{itemize}
section{Función objetivo}
\begin{center}
    \(\displaystyle \max\ \Pi \;=\; V^{post}_{i} \;-\; C^{Total}\)
\end{center}

\vspace{-2mm}
\noindent
\textbf{Donde:} \(C^{Total}=\) (costos de construcción, materiales, mano de obra, instalaciones, etc.).


% ===========================
% DOMINIOS
% ===========================
\subsection{Dominios}
\begin{align}
& Bedroom_i,\, FullBath_i,\, HalfBath_i,\, Kitchen_i,\, GarageCars_i \in \mathbb{Z}_{\ge 0},\\
& Floor1_i,\, Floor2_i,\, HasPool_i,\, HasWoodDeck_i,\, HasOpenPorch_i,\, HasEnclosedPorch_i,\, Has3SsnPorch_i,\, HasScreenPorch_i \in \{0,1\}.
\end{align}


% ===========================
% PARÁMETROS GLOBALES (con citas inmediatas)
% ===========================
\subsection{Parámetros globales con cotas en función del lote}
\label{sec:param-global}
\begin{align}
& U^{(1)}_i \;=\; \bar{\alpha}_1\, LotArea_i,\quad
  U^{(2)}_i \;=\; \bar{\alpha}_2\, LotArea_i,\quad
  U^{\text{plan}}_i \;=\; \bar{\alpha}_{\text{plan}}\, LotArea_i. \\
& \bar{\alpha}_1=0.6,\ \bar{\alpha}_2=0.5,\ \bar{\alpha}_{\text{plan}}=0.6.
\end{align}
\noindent\textit{\footnotesize Fuente: NAHB (2023), HUD (2020), IRC (2021). Valores de ocupación del lote del 50–65\% para vivienda unifamiliar y límites prudentes por piso/techo.}

\vspace{2mm}
\noindent
\textbf{Promedio de superficie por barrio}
\begin{align}
& GrLivArea_{i,z} \;\le\; \bar{A}^{prom}_z, \qquad \forall i,z.
\end{align}
\noindent\textit{\footnotesize Fuente: De Cock (2011) (Ames Housing), usado como cota empírica intra-barrio.}

\vspace{2mm}
\noindent
\textbf{Factor de superficie real de techo (pendiente/solapes)}
\begin{align}
& \gamma_{s,m}\ \ge 1 \qquad \forall s\in S,\ \forall m\in M.
\end{align}
\noindent\textit{\footnotesize Fuente: NAHB (2023), ARMA (2021), Roofing Alliance (2022).}

% ===========================
% RESTRICCIONES DE ÁREA Y CONSTRUCCIÓN
% ===========================
\subsection{Restricciones de Área y Construcción}
\begin{align}
& 1stFlrSF_{i} + TotalPorchSF_{i} + AreaPool_{i} \;\le\; LotArea_{i}, \quad \forall i,\\[2pt]
& 2ndFlrSF_{i} \;\le\; 1stFlrSF_{i}, \quad \forall i,\\[2pt]
& GrLivArea_{i,z} \;=\; 1stFlrSF_{i} + 2ndFlrSF_{i}, \quad \forall i.
\end{align}


% ===========================
% VARIABLES DE ÁREA POR AMBIENTE
% ===========================
\subsection{Variables de Área por Ambiente}
\textbf{Variables (todas \(\ge 0\)):}
\[
AreaKitchen_i,\ AreaFullBath_i,\ AreaHalfBath_i,\ AreaBedroom_i,\ 
AreaFoundation_i,\ AreaPool_i,\ AreaExterior1st_i,\ AreaExterior2nd_i,\ AreaMasonry_i.
\]
\textbf{Consistencias:}
\begin{align}
& TotalBsmtSF_i \;=\; BsmtFinSF1_i + BsmtFinSF2_i,\\
& TotalArea_i \;=\; 1stFlrSF_i + 2ndFlrSF_i + TotalBsmtSF_i.
\end{align}


% ===========================
% LÍMITES MÁXIMOS POR TIPO DE VIVIENDA
% ===========================
\subsection{Máximo de ambientes repetidos según tipo de vivienda}
\textbf{Parámetros (máximos):}
\begin{align}
& B_{\max}^{1Fam}=6,\ B_{\max}^{TwnhsE}=4,\ B_{\max}^{TwnhsI}=4,\ B_{\max}^{Duplx}=5,\ B_{\max}^{2FmCon}=8,\\
& F_{\max}^{1Fam}=4,\ F_{\max}^{TwnhsE}=3,\ F_{\max}^{TwnhsI}=3,\ F_{\max}^{Duplx}=4,\ F_{\max}^{2FmCon}=6,\\
& H_{\max}^{1Fam}=2,\ H_{\max}^{TwnhsE}=2,\ H_{\max}^{TwnhsI}=2,\ H_{\max}^{Duplx}=2,\ H_{\max}^{2FmCon}=3,\\
& K_{\max}^{1Fam}=1,\ K_{\max}^{TwnhsE}=1,\ K_{\max}^{TwnhsI}=1,\ K_{\max}^{Duplx}=2,\ K_{\max}^{2FmCon}=2,\\
& Ch_{\max}^{1Fam}=1,\ Ch_{\max}^{TwnhsE}=1,\ Ch_{\max}^{TwnhsI}=1,\ Ch_{\max}^{Duplx}=1,\ Ch_{\max}^{2FmCon}=2.
\end{align}
\noindent\textit{\footnotesize Fuente: De Cock (2011) (distribuciones observadas Ames) y criterios de escala funcional NAHB (2023).}

\textbf{Límites:}
\begin{align}
& Bedrooms_i \;\le\; \sum_b B_{\max}^{b} \, BldgType_{i,b},\\
& FullBath_i \;\le\; \sum_b F_{\max}^{b} \, BldgType_{i,b},\\
& HalfBath_i \;\le\; \sum_b H_{\max}^{b} \, BldgType_{i,b},\\
& Kitchen_i \;\le\; \sum_b K_{\max}^{b} \, BldgType_{i,b},\\
& FirePlaces_i \;\le\; \sum_b Ch_{\max}^{b} \, BldgType_{i,b}.
\end{align}


% ===========================
% MÍNIMOS FUNCIONALES DE ÁREA (con cita)
% ===========================
\subsection{Áreas mínimas por ambiente}
\begin{align}
& AreaFullBath_i \;\ge\; 40\cdot FullBath_i, \qquad
  AreaHalfBath_i \;\ge\; 20\cdot HalfBath_i,\\
& AreaBedroom_i \;\ge\; 70\cdot Bedroom_i, \qquad
  AreaKitchen_i \;\ge\; 75\cdot Kitchen_i.
\end{align}
\noindent\textit{\footnotesize Fuente: HUD (2020) y NAHB (2023) (superficies mínimas funcionales típicas).}


% ===========================
% PISOS Y TECHO (linealizado) + citas
% ===========================
\subsection{Pisos y techo: definición de áreas y dependencia estilo/material}
\textbf{Elección de pisos:}
\begin{align}
& Floor1_i + Floor2_i = 1.
\end{align}

\textbf{Área en planta a cubrir por el techo (linealización):}
\begin{align}
& PR1_i \le 1stFlrSF_i,\quad PR1_i \le U^{(1)}_i\, Floor1_i,\quad
  PR1_i \ge 1stFlrSF_i - U^{(1)}_i(1-Floor1_i),\quad PR1_i \ge 0,\\
& PR2_i \le 2ndFlrSF_i,\quad PR2_i \le U^{(2)}_i\, Floor2_i,\quad
  PR2_i \ge 2ndFlrSF_i - U^{(2)}_i(1-Floor2_i),\quad PR2_i \ge 0,\\
& PlanRoofArea_i = PR1_i + PR2_i,\qquad 0 \le PlanRoofArea_i \le U^{\text{plan}}_i.
\end{align}
\noindent\textit{\footnotesize Fuente: NAHB (2023), HUD (2020), IRC (2021) para proporciones máximas por piso y cubierta.}

\textbf{Selección estilo y material (one–hot):}
\begin{align}
& \sum_{s\in S} RoofStyle_{i,s} = 1,\qquad \sum_{m\in M} RoofMatl_{i,m} = 1.
\end{align}

\textbf{Conjunción estilo–material y área real:}
\begin{align}
& Y_{i,s,m} \le RoofStyle_{i,s},\quad Y_{i,s,m} \le RoofMatl_{i,m},\\
& Y_{i,s,m} \ge RoofStyle_{i,s} + RoofMatl_{i,m} - 1,\qquad
  \sum_{s,m} Y_{i,s,m}=1,\\[2pt]
& Z_{i,s,m} \le PlanRoofArea_i,\quad Z_{i,s,m} \le U^{\text{plan}}_i Y_{i,s,m},\\
& Z_{i,s,m} \ge PlanRoofArea_i - U^{\text{plan}}_i(1-Y_{i,s,m}),\quad Z_{i,s,m}\ge 0,\\
& ActualRoofArea_i \;=\; \sum_{s,m} \gamma_{s,m}\, Z_{i,s,m}.
\end{align}
\noindent\textit{\footnotesize Fuente: ARMA (2021), Roofing Alliance (2022) para \(\gamma_{s,m}\) (pendiente/solapes).}


% ===========================
% PORCHES (con citas)
% ===========================
\subsection{Restricciones de Porches}
\textbf{Parámetros (mínimos funcionales y cotas por tipo):}
\[
a_{\min}^{open}=40,\quad a_{\min}^{encl}=60,\quad a_{\min}^{3ssn}=80,\quad a_{\min}^{screen}=40.
\]
\[
U^{open}_i=0.10\,LotArea_i,\;\;
U^{encl}_i=0.10\,LotArea_i,\;\;
U^{3ssn}_i=0.10\,LotArea_i,\;\;
U^{screen}_i=0.05\,LotArea_i,\;\;
U^{porch,total}_i=0.25\,LotArea_i.
\]
\noindent\textit{\footnotesize Fuente: NAHB (2023) lineamientos residenciales; límites proporcionales de ocupación exterior.}

\textbf{Definición y activación:}
\begin{align}
& TotalPorchSF_i = OpenPorchSF_i + EnclosedPorch_i + 3SsnPorch_i + ScreenPorch_i,\\
& a_{\min}^{open} HasOpenPorch_i \le OpenPorchSF_i \le U^{open}_i HasOpenPorch_i,\\
& a_{\min}^{encl} HasEnclosedPorch_i \le EnclosedPorch_i \le U^{encl}_i HasEnclosedPorch_i,\\
& a_{\min}^{3ssn} Has3SsnPorch_i \le 3SsnPorch_i \le U^{3ssn}_i Has3SsnPorch_i,\\
& a_{\min}^{screen} HasScreenPorch_i \le ScreenPorch_i \le U^{screen}_i HasScreenPorch_i,\\
& TotalPorchSF_i \le U^{porch,total}_i,\qquad TotalPorchSF_i \le 1stFlrSF_i.
\end{align}
\noindent\textit{\footnotesize Fuente: NAHB (2023) para mínimos funcionales y compatibilidad estructural con primer piso.}

\textbf{Compatibilidad con deck y piscina:}
\begin{align}
& WoodDeckSF_i + TotalPorchSF_i + AreaPool_i \le 0.35\, LotArea_i,\qquad
  WoodDeckSF_i + OpenPorchSF_i \le 0.20\, LotArea_i.
\end{align}
\noindent\textit{\footnotesize Fuente: NAHB (2023) (capacidad conjunta de espacios exteriores).}


% ===========================
% SÓTANO SIN HasBasement (vía NA en exposición) + citas
% ===========================
\subsection{Restricciones de Sótano (sin $HasBasement$)}
\textbf{Parámetros:}
\[
U^{bsmt}_i = 0.5\,LotArea_i,\quad
U^{bF}=2,\quad U^{bH}=1,\quad
A^{fin}_{\min}=100\ \text{ft}^2.
\]
\noindent\textit{\footnotesize Fuente: IRC (2021), HUD (2020) (capacidad de sótano y baños), NAHB (2023) (mínimos funcionales de acabado).}

\textbf{Exposición (one–hot) \& existencia:}
\begin{align}
& \sum_{x\in\{Gd,Av,Mn,No,NA\}} BsmtExposure_{i,x} = 1,\\
& \sum_{x\in\{Gd,Av,Mn,No\}} BsmtExposure_{i,x} = 1 - BsmtExposure_{i,NA}.
\end{align}

\textbf{Capacidad, partición y activación por tipo:}
\begin{align}
& TotalBsmtSF_i \le U^{bsmt}_i\,(1-BsmtExposure_{i,NA}),\\
& BsmtFinSF1_i + BsmtFinSF2_i = TotalBsmtSF_i,\\
& BsmtFinSF1_i \le U^{bsmt}_i \!\!\sum_{b_1\neq NA} BsmtFinType1_{i,b_1},\quad
  BsmtFinSF2_i \le U^{bsmt}_i \!\!\sum_{b_2\neq NA} BsmtFinType2_{i,b_2},\\
& BsmtFinSF1_i \ge A^{fin}_{\min} \!\!\sum_{b_1\in\{\text{GLQ,ALQ,BLQ,Rec,LwQ}\}} \!\! BsmtFinType1_{i,b_1},\\
& BsmtFinSF2_i \ge A^{fin}_{\min} \!\!\sum_{b_2\in\{\text{GLQ,ALQ,BLQ,Rec,LwQ}\}} \!\! BsmtFinType2_{i,b_2}.
\end{align}

\textbf{Baños en sótano sólo con acabado real:}
\begin{align}
& BsmtFullBath_i \le U^{bF}\!\left(
\sum_{b_1\in\{\text{GLQ,ALQ,BLQ,Rec,LwQ}\}} \!\! BsmtFinType1_{i,b_1} +
\sum_{b_2\in\{\text{GLQ,ALQ,BLQ,Rec,LwQ}\}} \!\! BsmtFinType2_{i,b_2} \right),\\
& BsmtHalfBath_i \le U^{bH}\!\left(
\sum_{b_1\in\{\text{GLQ,ALQ,BLQ,Rec,LwQ}\}} \!\! BsmtFinType1_{i,b_1} +
\sum_{b_2\in\{\text{GLQ,ALQ,BLQ,Rec,LwQ}\}} \!\! BsmtFinType2_{i,b_2} \right).
\end{align}
\noindent\textit{\footnotesize Fuente: IRC (2021), HUD (2020), NAHB (2023).}


% ===========================
% GARAGE (con cita)
% ===========================
\subsection{Garage}
\textbf{Parámetros:}
\[
\overline{C}^{\text{cars}}=4,\qquad
\overline{A}^{\text{garage}}_i = 0.2\, LotArea_i.
\]
\noindent\textit{\footnotesize Fuente: De Cock (2011) (Ames Housing) para conteos; NAHB (2023) para proporción de área.}

\textbf{Capacidades básicas:}
\begin{align}
& GarageCars_i \le \overline{C}^{\text{cars}} \,(1 - GarageType_{i,NA}),\\
& GarageArea_i \le \overline{A}^{\text{garage}}_i \,(1 - GarageType_{i,NA}).
\end{align}
\textbf{Acabado sólo si hay garage:}
\begin{align}
& GarageFinish_{i,NA} = GarageType_{i,NA},\\
& GarageFinish_{i,\text{Fin}} + GarageFinish_{i,\text{RFn}} + GarageFinish_{i,\text{Unf}}
  = 1 - GarageType_{i,NA}.
\end{align}


% ===========================
% BIBLIOGRAFÍA MÍNIMA
% ===========================
\subsection*{Fuentes}
\begin{itemize}\setlength\itemsep{0pt}
\item International Code Council (2021). \textit{International Residential Code (IRC 2021).}
\item U.S. HUD (2020). \textit{Minimum Property Standards (Handbook 4910.1).}
\item NAHB (2023). \textit{Residential Construction Guidelines.}
\item ARMA (2021). \textit{Residential Asphalt Roofing Manual.}
\item Roofing Alliance (2022). \textit{Technical Guide to Roof System Performance.}
\item De Cock, D. (2011). \textit{Ames, Iowa: Alternative to the Boston Housing Data.}
\end{itemize}

\end{document}
