\section{Supuestos Construcción}
\label{sec:supuestos construcción}
A continuación se detallan supuestos del modelo de construcción.\\
Con el fin de que se entienda mejor, las restricciones se van explicando por variable. Se utilizó como ayuda la IA Chat GPT para implementar las restricciones generalizadas a cada caso específico, el detalle se enseña en el siguiente link: https://chatgpt.com/g/g-p-68ab9cccece88191b4d3995995422cc7/c/68f28f5b-82bc-8329-8d63-c87aa5e59d82
{Calidad de Construcción}\\
{Supuesto 1:} Todas las construcciones se consideran de calidad excelente por ser de nueva edificación. Las variables de calidad y condición toman su valor máximo porque se asume que el nuevo material está en su mejor estado.
\begin{itemize}
    \item $\texttt{OverallQual} = 10$ (Very Excellent)
    \item $\texttt{OverallCond} = 10$ (Very Excellent)
    \item $\texttt{ExterQual} = \text{Ex}$ (Excellent)
    \item $\texttt{ExterCond} = \text{Ex}$ (Excellent)
    \item $\texttt{BsmtQual} = \text{Ex}$ (Excellent)
    \item $\texttt{HeatingQC} = \text{Ex}$ (Excellent)
    \item $\texttt{KitchenQual} = \text{Ex}$ (Excellent)
\end{itemize}

{Exclusividad en Sistemas}\\
{Supuesto 2:} Solo puede seleccionarse una opción en variables categóricas de sistemas y características estructurales.
\begin{itemize}
    \item $\texttt{Electrical}$: Solo un tipo de sistema eléctrico
    \item $\texttt{Heating}$: Solo un tipo de sistema de calefacción
    \item $\texttt{RoofStyle}$: Solo un estilo de techo
    \item $\texttt{RoofMatl}$: Solo un material de techo
    \item $\texttt{HouseStyle}$: Solo un estilo de vivienda
    \item $\texttt{Foundation}$: Solo un tipo de cimentación
\end{itemize}

{Completitud de Áreas}\\
{Supuesto 3:} No existen áreas sin terminar en la construcción. Todas las áreas deben estar completamente finalizadas.
\begin{itemize}
    \item $\texttt{BsmtFinType1} \neq \text{Unf}$ (No unfinished basement)
    \item $\texttt{BsmtFinType2} \neq \text{Unf}$ (No unfinished basement)
    \item $\texttt{BsmtUnfSF} = 0$ (No unfinished basement area)
    \item $\texttt{LowQualFinSF} = 0$ (No low quality finished area)
    \item $\texttt{HouseStyle} \notin \{\text{1.5Unf}, \text{2.5Unf}\}$ (No unfinished levels)
\end{itemize}

{Límite de Pisos}\\
{Supuesto 4:} Las viviendas pueden tener máximo 2 pisos completos. No se consideran niveles parciales o medios pisos sin terminar.
\begin{itemize}
    \item $\texttt{HouseStyle} \in \{\text{1Story}, \text{2Story}\}$
    \item $\texttt{2ndFlrSF} \leq \texttt{1stFlrSF}$
    \item No se permiten estilos con niveles parciales (1.5Fin,\\ 1.5Unf, 2.5Fin, 2.5Unf)
\end{itemize}

{Parámetros Fijos del Terreno}\\
{Supuesto 5:} Las características físicas del terreno son parámetros fijos que no pueden modificarse.
\begin{itemize}
    \item $\texttt{LotArea}$: Parámetro fijo (área total del terreno)
    \item $\texttt{LotFrontage}$: Parámetro fijo (frente lineal de la propiedad)
    \item $\texttt{Street}$: Parámetro fijo (tipo de acceso vial)
    \item $\texttt{Alley}$: Parámetro fijo (tipo de acceso por callejón)
    \item $\texttt{LotShape}$: Parámetro fijo (forma del terreno)
    \item $\texttt{LandContour}$: Parámetro fijo (topografía del terreno)
    \item $\texttt{LotConfig}$: Parámetro fijo (configuración del lote)
    \item $\texttt{LandSlope}$: Parámetro fijo (pendiente del terreno)
\end{itemize}


{Límites Proporcionales al Terreno y tipo de casa}\\
{Supuesto 8:} Los límites de construcción se basan en proporciones del área total del terreno (LotArea).
\begin{itemize}
    \item \textbf{Área construida:} Máximo 80\% del terreno para área habitable
    \item \textbf{Pisos:} Primer piso máximo 60\%, segundo piso máximo 50\% del terreno
    \item \textbf{Sótano:} Máximo 50\% del área del terreno
    \item \textbf{Garaje:} Máximo 20\% del terreno
    \item \textbf{Piscina y porches:} Entre 5-15\% del terreno según tipo.
\end{itemize}

Para las cantidades máximas y mínimas de cantidad de baños, dormitorios, etc. se definen cotas mínimas y máximas obtenidas de información bilbiográfica.\\
{Área Mínima para Deck de Madera}\\
{Supuesto 9:} Si se construye un deck de madera, debe tener un área mínima de 40 ft² para ser funcional, con funcional nos referimos a que pueda caer una mesa y al menos 4 sillas.
\begin{itemize}
    \item Área mínima: 40 ft² (approx. 2m x 2m) para mobiliario básico
    \item Área máxima: 15\% del terreno para mantener espacio útil en patio
    \item Opcional: La vivienda puede o no tener deck de madera
\end{itemize}
{Forma geométrica de la vivienda}\\
{Supuesto 10:} La forma del primer y segundo piso será rectangular.\\
{Utilities}\\
{Supuesto 11:} Se considerará que la vivienda construida tendrá todas las utilities (All Public Utilities)
\begin{center}
    $Utilities_{i,AllPub}=1, \hspace{5mm} \forall i$\\
    $Utilities_{i,u}=0 \hspace{5mm} u \in\{NoSewr,NoSeWa,ELO\}$
\end{center}

\section{Costos construcción}
$C_{i}^{Total}=C_{i}^{Foundation}+C_i^{Roof}+C_{i}^{Heating}+C_{i}^{CentralAir}+C_{i}^{Electrical}+C_{i}^{PavedDrive}+C_{i}^{Kitchen}+C_{i}^{HalfBaths}+C_{i}^{FullBaths}+C_{i}^{Bedroom}+C_{i}^{Garage}+C_{i}^{Porch}+C_{i}^{WoodDeck}+C_{i}^{Reja} + C{i}^{Basement}+C_{i}^{MasVnr}+C_{i}^{Exterior}+C_{i}^{MiscFeature} +C_{i}^{FirePlaces}$\\
Donde:
\begin{itemize}
    \item $C_{i}^{Foundation}=\sum _{f} L_{i,f}\cdot C_{f}$
    \item $C_i^{Roof}= \sum_{s} \sum_{m} C_{m}\cdot (\gamma_{s,m}\cdot Z_{i,s,m})$
    \item $C_{i}^{Heating}=\sum_{h} C_{h,Ex}\cdot HasHeating_{i,h}$
    \item $C_{i}^{CentralAir}= \sum_{a} C_{i,a}$
    \item $C_{i}^{Electrical}=\sum_{e} C_{e} \cdot Electrical_{i,e}$
    \item $C_{i}^{PavedDrive}= \sum_{d} C_{d}\cdot PavedDrive_{i,d}$
    \item $C_{i}^{Kitchen}=\sum_{k}C_{k,Ex}\cdot AreaKitchen_{i}$
    \item $C_{i}^{HalfBaths}= C_{HalfBath}\cdot AreaHalfBath_{i}$
    \item $C_{i}^{FullBaths}= C_{FullBath}\cdot AreaFullBath_{i}$
    \item $C_{i}^{Bedroom}= C_{Bedroom}\cdot AreaBedroom_{i}$
    \item $C_{i}^{Garage}=C_{Garage}\cdot GarageArea_{i}$
    \item $C_{i}^{Porch}= OpenPorch_{i}\cdot C_{OpenPorch}+EnclosedPorch_{i}\cdot C_{EnclosedPorch}+3SsnPorch_{i}\cdot C_{3SsnPorch}+ScreenPorch_{i} \cdot C_{ScreenPorch}$
    \item $C_{i}^{WoodDeck}=WoodDesk\cdot C_{WoodDeck}$
    \item $C_{i}^{Fence}=F_{i}\cdot C_{Fence}$
    \item $C_{i}^{Exterior}=C_{e1} \cdot AreaExterior1st_{i,e1}$
    \item $C_i^{MasVnr}=\sum _{t}MvProd_{i,t}\cdot C_{t}$
    \item $C_{i}^{Garage}= \sum_{g} GA_{i,g}\cdot C_{garage}$
    \item $C_{i}^{FirePlaces}=C_{f,Ex}\cdot FirePlaces_{i}$
    \item $C_{i}^{Basement}= C_{g,Ex}\cdot AreaGarage_{i}$
\end{itemize}
\section{Variables Binarias: no presentes en la base de datos}
\begin{itemize}
    \item $Floor1_{i} \in {0,1} \hspace{5mm} \forall i \in \mathcal{I}$: Casa i tiene 1 piso
    \item $Floor2_{i} \in 0,1 \hspace{5mm} \forall i \in \mathcal{I}$: Casa tiene 2 pisos
    \item $HasFence_{i} \in 0,1$ : 1 si vivienda i tiene reja, 0 eoc
\end{itemize}
\section{Variables de área: Como la casa se construye desde cero es necesario asegurar que sea funcinal y que las áreas construidas sean coherentes con el total de áreas construidas.}
Por ejemplo, que exista área disponible para sala de estar o recreación.
\begin{itemize}
    \item $AreaBedrooms_{i} \in \mathbf{Z}_{\geq 0}$: Área total dormitorios dentro de la casa
    \item $AreaBedroom1_{i} \in \mathbf{Z_{\geq 0}}$: Área dormitorios primer piso.
    \item $AreaBedroom2_{i} \in \mathbf{Z_{\geq 0}}$: Área dormitorios en segundo piso.
    \item $AreaOtherRooms1_{i} \in \mathbf{Z_{\geq 0}}$: Área OtherRooms es primer piso.
    \item $AreaOtherRooms2_{i} \in \mathbf{Z_{\geq 0}}$: Área OtherRooms en segundo piso
    \item $AreaOther_{i}\in \mathbf{Z_{\geq 0}}$: Área Total OtherRooms dentro de la casa.
    \item $AreaKitchen_{i} \in \mathbf{Z}_{\geq 0}$: Área cocina en pies cuadrados
    \item $AreaFullBath_{i} \in \mathbf{Z}_{\geq 0}$: Área total FullBath dentro de la casa.
    \item $AreaHalfBath_{i} \in \mathbf{Z}_{\geq 0}$: Área total HalfBath dentro de la casa. 
    \item $AreaKitchen1_{i} \in \mathbf{Z}_{\geq 0}$: Área cocina primer piso
    \item $AreaKitchen2_{i}\in \mathbf{Z}_{\geq 0}$: Área kitchen en segundo piso
    \item $AreaFullBath1_{i}\in \mathbf{Z}_{\geq 0}$: Área total FullBath en primer piso.
    \item $AreaFullBath2_{i}\in \mathbf{Z}_{\geq 0}$: Área total FullBath en segundo piso
    \item $AreaHalfBath1_{i}\in \mathbf{Z}_{\geq 0}$: Area HalfBath en primer piso
    \item $AreaHalfBath2_{i}\in \mathbf{Z}_{\geq 0}$: Área total HalfBath en segundo piso
    \item $AreaFoundation_{i}\in \mathbf{Z}_{\geq 0}$: Área cimentación
    \item $AreaRoof_{i} \in \mathbf{Z}_{\geq 0}$: Área de techo.
    \item $PR1_i,\, PR2_i \ge 0$ \quad (\textit{auxiliares para linealizar $AreaRoof_i$})
    \item $PR1_{i}:$ Área del primer piso que se cubre con techo, si vivienda tiene 1 piso
    \item $PR2_{i}$ Area del segundo piso que se cubre con techo, si vivienda tiene 2 pisos
    \item $P_{i}^{(1)}$: Perímetro primer piso
    \item $P_{i}^{(2)}$: Perímetro segundo piso
    \item $W_{i,e1}:$ área exterior de la casa se cubre con el material $e_1$
    \item $GA_{i,g}$: área del garage tipo $g$
\end{itemize}
\section{Variables de conteo: Se utilizan para asegurar que existen cantidades funcionales de baños y piezas}
\begin{itemize}
    \item $FullBath1_{i}\in \mathbf{Z}_{\geq 0}$ cantidad de Full Baths en primer piso
    \item $FullBath2_{i}\in \mathbf{Z}_{\geq 0}$ cantidad de FullBath en segundo piso
    \item $HalfBath1_{i} \in \mathbf{Z}_{\geq 0}$: cantidad de HalfBath en primer piso
    \item $HalfBath2_{i} \in \mathbf{Z}_{\geq 0}$: cantidad de HalfBath en segundo piso.
    \item $Kitchen1_{i} \in \mathbf{Z}_{\geq 0}$: Cantidad de cocinas en primer piso
    \item $Kitchen2_{i} \in \mathbf{Z}_{\geq 0}$: cantidad cocinas en segundo piso
    \item $Bedroom1_{i} \in \mathbf{Z}_{\geq 0}$: cantidad de dormitorios en primer piso
    \item $Bedroom2_{i}\in \mathbf{Z}_{\geq 0}$: cantidad de dormitorios en segundo piso.
    \item $OtherRooms1_{i}\in \mathbf{Z}_{\geq 0}$: cantidad de otras habitaciones en el primer piso.
    \item $OtherRooms2_{i}\in \mathbf{Z}_{\geq 0}$: cantidad de otras habitaciones en segundo piso
    \item $OtherRooms_{i}\in \mathbf{Z}_{\geq 0}$: cantidad de otras habitaciones dentro de la casa
\end{itemize}
\section{Restricciones}
\begin{itemize}
    \item {Restricciones de exclusividad}
\begin{align}

    &\sum_{s} MSSubClass_{i,s}=1 \hspace{5mm} \forall i \in \mathcal{I} \\
    
    & \sum_{\mathcal{B}\in b} BldgType_{i,b}= 1 \hspace{5mm} \forall i \in \mathcal{I}\\
    
    & \sum _{hs} HouseStyle_{i,hs}=1\hspace{5mm} \forall i \in \mathcal{I}\\
    
    &\sum_{r} RoofStyle_{i,r}=1\hspace{5mm} \forall i \in \mathcal{I}\\
    
    &\sum_{m} RoofMatl_{i,m}=1\hspace{5mm} \forall i \in \mathcal{I}\\
    
    &\sum_{e1} Exterior1st_{i,e1}=1\hspace{5mm} \forall i \in \mathcal{I}\\
    
    &\sum_{e2} Exterior2nd_{i,e2}=1\hspace{5mm} \forall i \in \mathcal{I}\\
    
    &\sum_{t} MasVnrType_{i,t}=1\hspace{5mm} \forall i \in \mathcal{I}\\
    
    &\sum_{f} Foundation_{i,f}=1\hspace{5mm} \forall i \in \mathcal{I}\\
    
    &\sum_{x} BsmtExpoure_{i,x}=1\\
    
    &\sum_{b1} BsmtFinType1_{i,b1}=1\\
    
    &\sum_{b2} BsmtFinType2_{i,b2}=1\\
    
    &\sum_{h} Heating_{i,h}= 1\\
    
    &\sum_{a} CentralAir_{i,a}=1\\
    
    &\sum_{e} Electrical_{i,e}=1\\
    
    &\sum_{g} GarageType_{i,g}=1\\
    
    &\sum_{gf} GarageFinish_{i,gf}=1\\
    
    &\sum_{p} PavedDrive_{i,p}=1\\
    
    &\sum_{misc} MiscFeature_{i,misc}=1\\
 
\end{align}
    \item{Consistencia de Áreas}\\
    \item {Áreas construidas no pueden sobrepasar el área del terreno}\\
\begin{align}
    &1stFlrSF_{i} + TotalPorchSF_{i} + AreaPool_{i} \leq LotArea_{i}, \hspace{5mm}\forall i \in \mathcal{I} 
\end{align}
    \item {El segundo piso no puede ser más grande que el primero}
\begin{align}
    &2ndFlrSF_{i} \leq 1stFlrSF_{i} \hspace{5mm}\forall i \in \mathcal{I}
\end{align}
    \item {Area Habitable}
\begin{align}
    &GrLivArea_{i} = 1stFlrSF_{i} + 2ndFlrSF_{i}
\end{align}
    \item {Area Total FullBath  es igual al area baños 1er piso + area baños 2do piso}
\begin{align}
    &AreaFullBath_{i}=AreaFullBath1_{i}+ AreaFullBath2_{i}
\end{align}
    \item {Area Total HalfBath es igual al area baños 1er piso + Area baños 2do piso}
\begin{align}
    &AreaHalfBath_{i}=AreaHalfBath1_{i}+AreaHalfBath2_{i}
\end{align}
    \item{Tiene que haber un baño en el primer piso}
\begin{align}
    FullBath1_{i}\geq 1
\end{align}

    \item{Tiene que haber una cocina en el primer piso}
\begin{align}
    Kitchen1_{i}\geq 1
\end{align}

    \item {Areas Primer y segundo piso}
\begin{align}
    &2ndFlrSF_{i}\leq M_{max}^{2ndFlrSF} \cdot Floor2_{i}\\
    &2ndFlrSF_{i}\geq \epsilon \cdot Floor2_{i}\\
    &1stFlrSF_{i}\geq \epsilon ` \cdot (Floor1_{i}+Floor2_{i})
\end{align}\\
Donde:
    \begin{itemize}
        \item $\epsilon=450$
        \item $\epsilon `=350$
\end{itemize}
Se asume que los pisos de la vivienda cumplen con los mínimos habitables establecidos por el código residencial de Ames y los estándares del HUD y NAHB, fijando 450 ft² para el primer piso  \cite{decock2011} y 350 ft² para el segundo piso\cite{ HUD2022}\cite{NAHB2023}
    \item{Consistencia Cantidades}
    \item {Cantidad total de FullBaths}
    \[
FullBath_i \;=\; FullBath1_i + FullBath2_i \hspace{8mm} \forall i.
\]
    \item {Cantidad Total de HalfBath}
    \[
HalfBath_i \;=\; HalfBath1_i + HalfBath2_i \hspace{8mm} \forall i.
\]

    \item{Cantidad Total de Cocinas}
\[
Kitchen_i \;=\; Kitchen1_i + Kitchen2_i \hspace{8mm} \forall i.
\]

    \item {Máximo de repeticiones}
    \begin{itemize}
        \item Parámetros para Bedrooms según tipo de vivienda
        \begin{itemize}
            \item $Bed_{max}^{1Fam}=6$\\
            \item $Bed_{max}^{TwnhsE}=4$\\
            \item $Bed_{max}^{TwnhsI}=4$\\
            \item $Bed_{max}^{Dplx}=5$\\
            \item $Bed_{max}^{2FmCon}=8$
        \end{itemize}
    Los límites máximos de dormitorios por tipo de vivienda se calibraron en base a las distribuciones observadas en Ames Housing Dataset \cite{decock2011} y a las tipologías residenciales documentadas por la \cite {NAHB2023}
    \item {Máxima cantidad de habitaciones:}
\begin{align}
    Bedrooms_{i}\leq \sum_{b \in \mathcal{B}}Bed_{max}^{b} \cdot BldgType_{i,b} \hspace{5mm } \forall i \in \mathcal{I}
\end{align}
        \item  Parametros para FullBaths:
        \begin{itemize}
            \item $F_{max}^{1Fam}=4$\\
            \item $F_{max}^{TwnhsE}=3$\\
            \item $F_{max}^{TwnhsI}=3$\\
            \item $F_{max}^{Dplx}=4$\\
            \item $F_{max}^{2FmCon}=6$
        \end{itemize}
    Los límites máximos de baños completos por tipo de vivienda se establecieron en función de la distribución empírica observada en Ames Housing Dataset \cite{decock2011} y de las recomendaciones de diseño residencial de la \cite{NAHB2023} y el \cite{HUD2022}.
    Estos valores reflejan el estándar constructivo actual de viviendas de 1 a 2 familias en Ames, Iowa.
    \item {Máxima cantidad de FullBaths:}
\begin{align}
        FullBath_{i} \leq \sum_{b} F_{max}^{b} \cdot BldgType_{i,b}
\end{align}
        \item Parámetros HalfBath:
        \begin{itemize}
            \item $H_{max}^{1Fam}=2$\\
            \item $H_{max}^{TwnhsE}=2$\\
            \item $H_{max}^{TwnhsI}=2$\\
            \item $H_{max}^{Dplx}=2$\\
            \item $H_{max}^{2FmCon}=3$
        \end{itemize}
Los límites máximos de medios baños por tipo de vivienda se calibraron en base a la distribución observada en Ames Housing Dataset \cite{decock2011} y los estándares habitacionales del \cite {HUD2022} y la \cite{NAHB2023}.
    Estos límites aseguran coherencia con el espacio habitable y el tipo de edificación.
    \item {Máxima cantidad de HalfBaths:}
    \begin{align}
        HalfBath_{i} \leq \sum_{b} H_{max}^{b} \cdot BldgType_{i,b}
    \end{align}
        \item Parámetros Cocina:
        \begin{itemize}
            \item $K_{max}^{1Fam}=1$\\
            \item $K_{max}^{TwnhsE}=1$\\
            \item $K_{max}^{TwnhsI}=1$\\
            \item $K_{max}^{Dplx}=2$\\
            \item $K_{max}^{2FmCon}=2$
        \end{itemize}
    Los límites máximos de cocinas por tipo de vivienda se establecieron a partir de la distribución empírica de la variable KitchenAbvGr en el Ames Housing Dataset \cite{decock2011} y de las normas residenciales del \cite {HUD2022}) y la \cite{NAHB2023}.
    Se considera que viviendas unifamiliares y adosadas poseen una cocina única, mientras que dúplex y conversiones bifamiliares pueden incluir dos cocinas, una por unidad independiente.
    \item {Máxima cantidad de Cocinas:}
    \begin{align}
        Kitchen_{i}\leq \sum _{b} K_{max}^{b} BldgType_{i,b}
    \end{align}
        \item Parámetros Chimenea:
        \begin{itemize}
            \item $Ch_{max}^{1Fam}=1$\\
            \item $Ch_{max}^{TwnhsE}=1$\\
            \item $Ch_{max}^{TwnhsI}=1$\\
            \item $Ch_{max}^{Dplx}=1$\\
            \item $Ch_{max}^{2FmCon}=2$ 
    \end{itemize}
    El número máximo de chimeneas permitidas por tipo de vivienda se determinó a partir de la distribución de la variable Fireplaces del Ames Housing Dataset \cite{decock2011} y los estándares de construcción residencial establecidos por la \cite{NAHB2023}.
Se considera que las viviendas unifamiliares y adosadas pueden incluir a lo sumo una chimenea, mientras que las viviendas bifamiliares pueden tener dos, una por unidad independiente.
    \item{Máxima cantidad de chimeneas:}
    \begin{align}
        FirePlaces_{i} \leq \sum_{b} Ch_{max}^{b} BldgType_{i,b}
    \end{align}
\end{itemize}
    \item{Casa solo puedo tener 1 ó 2 pisos}
\begin{align}
    &Floor1_{i} +Floor2_{i}=1 \hspace{5mm} \forall i \in \mathcal{I}
\end{align}
    \item{Garage}
    \item {Consistencia Areas Garage}
\begin{align}
    150\cdot GarageCars_{i} \leq GarageArea_{i}\leq 250 \cdot GarageCars_{i}
\end{align}
    Esta relación garantiza una superficie mínima y máxima razonable por vehículo, en línea con los estándares de diseño residencial de \cite{NAHB2023} y \cite{ICC2021}).
    \item {Existencias de activación}
\begin{align}
& GarageCars_i \le \overline{C}^{\text{cars}} \,\big(1 - GarageType_{i,NA}\big) \qquad \forall i,\\
& GarageArea_i \le \overline{A}^{\text{garage}}_i \,\big(1 - GarageType_{i,NA}\big) \qquad \forall i.
\end{align}
Donde:
    \begin{itemize}
        \item $\bar{C}_{i}^{cars}=4$
        \item $\bar{A}_{i}^{garage}=0.2 LotArea_{i}$
    \end{itemize}
    \item {Acabados}
\begin{align}
& GarageFinish_{i,NA} = GarageType_{i,NA} \qquad \forall i,\\
& GarageFinish_{i,\text{Fin}} + GarageFinish_{i,\text{RFn}} 
  = 1 - GarageType_{i,NA} \qquad \forall i.
\end{align}


    \item {Mínimo Funcional}
\begin{align}
& GarageCars_i \ge 1 - GarageType_{i,NA} \qquad \forall i.
\end{align}
    \item {Cerca}
\begin{itemize}
    \item Parámetros:
    \begin{itemize}
        \item $L^{Reja}_{i}=LotFrontage_{i}$
    \end{itemize}
\end{itemize}

    \item {Area Techo}


\begin{align}
    &PR1_{i} \leq 1stFloorSF_{i}\\
    &PR1_{i} \leq U_{i}^{(1)}\cdot Floor1_{i}\\
    &PR1_{i} \geq 1stFloorSF_{i}-U_{i}^{(1)}\cdot (1-Floor_{i})\\
    &PR2_{i} \leq 2ndFloorSF_{i},\\
    &PR2_{i} \leq U_{i}^{(2)}\cdot Floor2_{i}\\
    &PR2_{i} \geq 2ndFloorSF_{i}-U_{i}^{(2)}\cdot (1-Floor2_{i})\\
\end{align}
\\
Donde:
    \begin{itemize}
        \item $U_{i}^{(1)}\geq 1stFlrSF_{i}$
        \item $U_{i}^{(2)}\geq 2ndFlrSF_{i}$
        \item $U_{i}^{plan} \geq max\{1stFlrSF_{i},2ndFlrSF_{i}\}$: cota superior
    \end{itemize}
\\
    \item {Área que debe ser cubierta}\\
\begin{align}
    PlanRoofArea_{i}=PR1_{i}+PR2_{i}\\
\end{align}
\item{Área real de techo que se debe construir de acuerdo a pendiente del tipo de techo}\\
\begin{align}
    ActualRoofArea_{i}=\sum_{s}\sum_{m} \gamma_{s,m} \cdot Z_{i,s,m}\\
\end{align}
\\
Donde:
    \begin{itemize}
        \item $\gamma_{s,m} \geq 1$. Factor de expansión según estilo
        \item  $Z_{i,s,m}$ variable auxiliar que linealiza 
        \item $Y_{i,s,m} \in 0,1$ toma valor 1 cuando se selecciona el estilo s y material m de la casa i
\end{itemize}


    \item {Combinacion estilo y material:Exclusividad}

\begin{align}
& Y_{i,s,m} \le RoofStyle_{i,s} \qquad \forall s\in S,\ \forall m\in M, \\[2pt]
& Y_{i,s,m} \le RoofMatl_{i,m} \qquad \forall s\in S,\ \forall m\in M, \\[2pt]
& Y_{i,s,m} \ge RoofStyle_{i,s} + RoofMatl_{i,m} - 1 \qquad \forall s\in S,\ \forall m\in M, \\[4pt]
& \sum_{s}\sum_{m} Y_{i,s,m} = 1.
\end{align}
    \item {Restricciones lineales de techo}

\begin{align}
& Z_{i,s,m} \le PlanRoofArea_i \qquad \forall s,m, \\[2pt]
& Z_{i,s,m} \le U^{\text{plan}}_i \, Y_{i,s,m} \qquad \forall s,m, \\[2pt]
& Z_{i,s,m} \ge PlanRoofArea_i - U^{\text{plan}}_i \,(1 - Y_{i,s,m}) \qquad \forall s,m, \\[2pt]
& Z_{i,s,m} \ge 0 \qquad \forall s,m, \\[6pt]
\end{align}

\cite{Dawid2023}


\item{Consistencias Áreas Globales}
\begin{align}

& TotalBsmtSF_i = BsmtFinSF1_i + BsmtFinSF2_i,\\
& TotalArea_i = 1stFlrSF_i + 2ndFlrSF_i + TotalBsmtSF_i
\end{align}

\item {Límites de ocupación}
\begin{align}
& 1stFlrSF_i \le M_{max}^{1stFlrSF},\quad \\2ndFlrSF_i \le M_{max}^{2ndFlrSF},\\
& TotalBsmtSF_i \le M_{max}^{TotalBasmt},\\ \quad 
& GarageArea_i \le M_{max}^{GarageArea}
\end{align}
\\
Donde:
    \begin{itemize}
        \item $M_{max}^{1stFlrSF}= 0.6LotArea$
        \item $M_{max}^{2ndFlrSF}=0.5 LotArea$
        \item $M_{max}^{TotalBasmt}= 0.5LotArea$
        \item $M_{max}^{GarageArea}= 0.2LotArea$
    \end{itemize}
    \item{Baños por cada Dormitorio}
\begin{align}
& 3\,FullBath_i \ge 2\,Bedroom_i
\end{align}

    \item {Piscina}\\
    \item {El área de la piscina tiene que acotarse al espacio que queda}

\begin{align}
AreaPool_i \le \Big(LotArea_i - 1stFlrSF_i - GarageArea_i - WoodDeckSF_i - OpenPorchSF_i - \\EnclosedPorch_i - ScreenPorch_i - 3SsnPorch_i\Big)\cdot HasPool_i,\\
AreaPool_i \le U_{max}^{Pool}\cdot HasPool_i,\qquad\\
AreaPool_i \ge U_{min}^{Pool} HasPool_i,\qquad\\
AreaPool_i \ge 0
\end{align}
\\
Donde:
    \begin{itemize}
        \item $U_{min}^{Pool}=160$
        \item $U_{max}^{Pool}= 0.1 LotArea$
    \end{itemize}
\cite{NAHB2023}

    \item {Porch}\\
    \item {Área total del Porch es la suma de todos los Porch}
\begin{align}
& TotalPorchSF_i = OpenPorchSF_i + EnclosedPorch_i + ScreenPorch_i + 3SsnPorch_i,\\
\\
& TotalPorchSF_i \le U_{max}^{TotPorch},\\
\\
& TotalPorchSF_i \le 1stFlrSF_i\\
\end{align}
\\
Donde: 
    \begin{itemize}
        \item $U_{max}^{TotPorch} \leq 0.25 LotArea$\\

    \item {Mínimos funcionales por tipo (activados por las binarias que ya declaraste)}\\
    \end{itemize}
 
\begin{align}
& OpenPorchSF_i \ge 40 \cdot HasOpenPorch_i,\\
\\
& EnclosedPorch_i \ge 60 \cdot HasEnclosedPorch_i,\\
\\
& ScreenPorch_i \ge 40 \cdot HasScreenPorch_i,\\
\\
& 3SsnPorch_i \ge 80 \cdot Has3SsnPorch_i\\
\end{align}

    \item {Compatibilidad de espacios exteriores}\\
\begin{align}
& WoodDeckSF_i + TotalPorchSF_i + AreaPool_i \le U_{max}^{AreaExt},\\
& WoodDeckSF_i + OpenPorchSF_i \le U_{max}^{WDyPorch}
\end{align}\\
Donde:\\
    \begin{itemize}
        \item $U_{max}^{AreaExt}=0.35LotArea$
        \item $U_{max}^{WDyPorch}=0.2LotArea$
    \end{itemize}
    \item {Deck}\\
\begin{align}
& U_{min}^{WD} \cdot HasWoodDeck_i \leq WoodDeckSF_i \leq  U_{max}^{WD}\,LotArea_i \cdot HasWoodDeck_i\\
\end{align}
Donde:\\
    \begin{itemize}
        \item $U_{min}^{WD}=40$ valor mínimo de WoodDeck
        \item $U_{max}^{WD}=0.15LotArea$: valor máximo de WoodDeck
    \end{itemize}

    \item {Acabados}
\begin{align}
& TotalBsmtSF_i \;\le\; 0.5\,LotArea_i \,\big(1 - BsmtExposure_{i,NA}\big). \label{eq:bsmt-cap}
\end{align}

    \item Partición de áreas terminadas
\begin{align}
& BsmtFinSF1_i + BsmtFinSF2_i \;=\; TotalBsmtSF_i. 

\end{align}

\begin{align}
BsmtFinSF1_i \;\le\; 0.5\,LotArea_i \!\!\!\sum_{b_1\in \Buno\setminus\{\text{NA}\}}\!\! BsmtFinType1_{i,b_1}, \label{eq:fin1-on}\\
BsmtFinSF2_i \;\le\; 0.5\,LotArea_i \!\!\!\sum_{b_2\in \Buno\setminus\{\text{NA}\}}\!\! BsmtFinType2_{i,b_2}. \label{eq:fin2-on}
\end{align}

\begin{align}
BsmtFinSF1_i \;\ge\; A^{fin}_{\min} \!\!\sum_{b_1\in \Bfin}\! BsmtFinType1_{i,b_1}, \qquad
  BsmtFinSF2_i \;\ge\; A^{fin}_{\min} \!\!\sum_{b_2\in \Bfin}\! BsmtFinType2_{i,b_2}. \label{eq:fin-min}
\end{align}


\begin{align}
BsmtFullBath_i \;\le\; 2 \left(
     \sum_{b_1\in \Bfin}\! BsmtFinType1_{i,b_1}
   + \sum_{b_2\in \Bfin}\! BsmtFinType2_{i,b_2} \right), \label{eq:bbath-full}\\
BsmtHalfBath_i \;\le\; 1 \left(
     \sum_{b_1\in \Bfin}\! BsmtFinType1_{i,b_1}
   + \sum_{b_2\in \Bfin}\! BsmtFinType2_{i,b_2} \right). \label{eq:bbath-half}
\end{align}

\begin{align}
BsmtFinSF1_i \;\le\; 0.5\,LotArea_i \,\big(1 - BsmtExposure_{i,NA}\big), \\
BsmtFinSF2_i \;\le\; 0.5\,LotArea_i \,\big(1 - BsmtExposure_{i,NA}\big), \\
BsmtFullBath_i \;\le\; 2\,\big(1 - BsmtExposure_{i,NA}\big), \qquad\\
  BsmtHalfBath_i \;\le\; 1\,\big(1 - BsmtExposure_{i,NA}\big).
\end{align}



    \item {Exterior}
\begin{align}
& \sum_{e_1} Exterior1st_{i,e_1} = UseExterior1st_i \qquad (\text{si decides usar }UseExterior1st_i),\\
& \sum_{e_2} Exterior2nd_{i,e_2} = UseExterior2nd_i,\\
\end{align}
    \item {Mamposteria}
    \item{Cota superior Mampostería}
\begin{align}
    MasVnrArea_{i}\leq U_{i}^{mas}
\end{align}\\
Donde:
    \begin{itemize}
        \item $U_{i}^{mas}=f_{max}^{mas}\cdot AreaExterior_{i}$
        \item $f_{max}^{mas}=0.4$
    \end{itemize}
\begin{align}
    &MasVnrArea_{i}\geq A_{min}^{MasVnr}\cdot (1-MasVnrType_{i,None})
\end{align}\\
Variable auxiliar:\\
\begin{align}
    &MvProd_{i,t}\leq MasVnrArea_{i}\\
    &MvProd_{i,t}\equiv MasVnrArea_{i} \cdot MasVnrType_{i,t}\\
    &MvProd_{i,t} \leq U_{i}^{mas}\cdot MasVnrType_{i,t}\\
    &MvProd_{i,t}\geq MasVnrArea_{i} - U_{i}^{mas}\cdot (1-MasVnrType_{i,t})\\
    &MvProd_{i,t}\geq 0
\end{align}

Donde:\\
\begin{align}

& MasVnrArea_i \le TotalArea_i,\\ \quad MasVnrArea_i \ge 0
\end{align}\\
Donde:\\
    \begin{itemize}
        \item $A_{min}^{MasVnr}=20 $ft, Cota inferior Area mampostería
        \item $A_{max}^{MasVnr}=2000$
    \end{itemize}


    \item {Garage}

\begin{align}
& GarageCars_i \le \overline{C}^{\text{cars}} \,\big(1 - GarageType_{i,NA}\big) \qquad \forall i,\\
& GarageArea_i \le \overline{A}^{\text{garage}}_i \,\big(1 - GarageType_{i,NA}\big) \qquad \forall i.
\end{align}
\begin{align}\\
& GarageCars_i \ge 1 - GarageType_{i,NA} \qquad \forall i.
\end{align}\\
\begin{align}
& GarageFinish_{i,NA} = GarageType_{i,NA} \qquad \forall i,\\
& GarageFinish_{i,\text{Fin}} + GarageFinish_{i,\text{RFn}} 
  = 1 - GarageType_{i,NA} \qquad \forall i.
\end{align}
Donde:
    \begin{itemize}
        \item $\bar{C}_{i}^{cars}=4$
        \item $\bar{A}_{i}^{garage}=0.2 LotArea$
    \end{itemize}
\begin{align}
    &GarageType_{i,NoAplica}=GarageFinish_{i,NoAplica}
\end{align}\\
Ahora es necesario linealizar para calcular los costos:
\\
\begin{align}
    GA_{i,g}\leq GarageArea_{i}\\
    GA_{i,g}\leq \bar A_{i}^{Garage}\cdot GarageType_{i,g}\\
    GA_{i,g}\geq GarageArea_{i}-\bar A_{i}^{Garage}(1-GarageType_{i,g})\\
    GA_{i,g}\geq 0
\end{align}

    \item{Basement}
    \item {Parámetros:}
\[
U^{bsmt}_i = 0.5\cdot LotArea_i,\qquad
U^{bF}=2,\qquad
U^{bH}=1,\qquad
A^{fin}_{\min}\ge 0 \;.
\]
    \item {Capacidad Máxima del Sótano}

\begin{align}
    &BsmtFinSF1_{i}+BsmtFinSF2_{i}=TotalBsmtSF_{i}\\
\end{align}\\
\\
\begin{align}
    &TotalBsmtSF_{i}\leq 0.5 LotArea_{i}(1-BsmtExpoure_{i,NA})\\
\end{align}\\


    \item {Existencia de sótano vía exposición \texttt{NA}:}\\
\begin{align}
& TotalBsmtSF_i \;\le\; U^{bsmt}_i \big(1 - BsmtExposure_{i,NA}\big). \label{eq:b-exist}
\end{align}\\
Donde:
    \begin{itemize}
        \item $U_{i}^{bsmt}=0.5LotArea$
    \end{itemize}

    \item {Activadores (definición auxiliar):}
\[
\phi^{(1)}_i \;=\; \sum_{b_1\in B_1\setminus\{\text{NA}\}} BsmtFinType1_{i,b_1}, \qquad
\phi^{(2)}_i \;=\; \sum_{b_2\in B_2\setminus\{\text{NA}\}} BsmtFinType2_{i,b_2},
\]
\[
\psi^{(1)}_i \;=\; \sum_{b_1\in \{\text{GLQ, ALQ, BLQ, Rec, LwQ}\}} BsmtFinType1_{i,b_1}, \qquad \\
\\ \psi^{(2)}_i \;=\; \sum_{b_2\in \{\text{GLQ, ALQ, BLQ, Rec, LwQ}\}} BsmtFinType2_{i,b_2}.
\]
Donde:
    \begin{itemize}
        \item $\phi_{i}^{(1)}, \phi_{i}^{(2)}$ indicadores de existencia de cualquier tipo de acabado menos NA
        \item $\psi _{i}^{(1)}, \psi_{i}^{(2)}$: indicadores de existencia de acabado real.
    \end{itemize}

    \item {Activación y mínimos de acabados por canal:}\\
\begin{align}
& BsmtFinSF1_i \;\le\; U^{bsmt}_i\, \phi^{(1)}_i, \qquad\\
  BsmtFinSF2_i \;\le\; U^{bsmt}_i\, \phi^{(2)}_i, \label{eq:b-on}\\
& BsmtFinSF1_i \;\ge\; A^{fin}_{\min}\, \psi^{(1)}_i, \qquad\\
  BsmtFinSF2_i \;\ge\; A^{fin}_{\min}\, \psi^{(2)}_i. \label{eq:b-min}
\end{align}\\

    \item {Baños en sótano sólo si hay acabado real:}\\
\\
\begin{align}
& BsmtFullBath_i \;\le\; U^{bF}\,\big(\psi^{(1)}_i + \psi^{(2)}_i\big), \qquad \\
  BsmtHalfBath_i \;\le\; U^{bH}\,\big(\psi^{(1)}_i + \psi^{(2)}_i\big). \label{eq:b-baths}
\end{align}

    \item {Apagado completo si \texttt{BsmtExposure\_NA}=1:}\\
\begin{align}
& BsmtFinSF1_i \;\le\; U^{bsmt}_i \big(1 - BsmtExposure_{i,NA}\big), \qquad \\
  BsmtFinSF2_i \;\le\; U^{bsmt}_i \big(1 - BsmtExposure_{i,NA}\big), \\
& BsmtFullBath_i \;\le\; U^{bF}\big(1 - BsmtExposure_{i,NA}\big), \qquad \\
  BsmtHalfBath_i \;\le\; U^{bH}\big(1 - BsmtExposure_{i,NA}\big). \label{eq:b-off}
\end{align}
    \begin{itemize}
        \item $U^{\text{bF}}\in\mathbb{Z}_{\ge 0}$: cota superior de baños completos en sótano (p.ej. $2$).
        \item $U^{\text{bH}}\in\mathbb{Z}_{\ge 0}$: cota superior de medios baños en sótano (p.ej. $1$).
        \item $A^{\text{fin}}_{\min}\ge 0$: área mínima funcional para declarar un acabado “real” (p.ej. $100~\text{ft}^2$).
    \end{itemize}

\begin{align}

& \sum_{x\in \{\text{Gd,Av,Mn,No}\}} BsmtExposure_{i,x}
   \;=\; 1 - BsmtExposure_{i,NA}. \label{eq:exp-exist}
\end{align}\\
\\

\begin{align}
& TotalBsmtSF_i \;\le\; U_{i}^{\text{bsmt}}\,\big(1 - BsmtExposure_{i,NA}\big). \label{eq:bsmt-cap}
\end{align}

\begin{align}
& BsmtFinSF1_i + BsmtFinSF2_i \;=\; TotalBsmtSF_i. \label{eq:bsmt-sum}
\end{align}

\begin{align}
& BsmtFinSF1_i \;\ge\; A^{\text{fin}}_{\min}\; \sum_{b_1\in \Bfin} BsmtFinType1_{i,b_1}, 
\qquad
  BsmtFinSF2_i \;\ge\; A^{\text{fin}}_{\min}\; \sum_{b_2\in \Bfin} BsmtFinType2_{i,b_2}. \label{eq:fin-min}
\end{align}

\begin{align}
& BsmtFullBath_i \;\le\; U^{\text{bF}} \left(
     \sum_{b_1\in \Bfin} BsmtFinType1_{i,b_1}
   + \sum_{b_2\in \Bfin} BsmtFinType2_{i,b_2} \right), \label{eq:bbath-full} \\[6pt]
& BsmtHalfBath_i \;\le\; U^{\text{bH}} \left(
     \sum_{b_1\in \Bfin} BsmtFinType1_{i,b_1}
   + \sum_{b_2\in \Bfin} BsmtFinType2_{i,b_2} \right). \label{eq:bbath-half}
\end{align}

\begin{align}
& BsmtFinSF1_i \;\le\; \phi^{\text{bsmt}}\; LotArea_i \,\big(1 - BsmtExposure_{i,NA}\big), \\[6pt]
& BsmtFinSF2_i \;\le\; \phi^{\text{bsmt}}\; LotArea_i \,\big(1 - BsmtExposure_{i,NA}\big), \\[6pt]
& BsmtFullBath_i \;\le\; U^{\text{bF}}\,\big(1 - BsmtExposure_{i,NA}\big), 
\qquad
  BsmtHalfBath_i \;\le\; U^{\text{bH}}\,\big(1 - BsmtExposure_{i,NA}\big).
\end{align}

    \item {Forzar Material Principal}
\begin{align}
    &Exterior1st_{i}=Exterior2nd_{i}
\end{align}


    \item {Foundation}
Si la cimentación es de madera o Loza no puede tener sótano segpun expertos\\
\begin{align}
    Foundation_{i,Slab}\leq BsmtExposure_{i,NA}\\
    Foundation_{i,Wood}\leq BsmtExposure_{i,NA}\\
\end{align}

\begin{align}
    AreaFoundation_{i}=1stFlrSF_{i}\\
    
\end{align}
Hacemos linealización:\\

\\
\begin{align}
    &FA_{i,f}\leq AreaFoundation_{i}\\
    &FA_{i,f}\leq U_{i}^{Found}\cdot Foundation_{i,f}\\
    &FA_{i,f}\geq AreaFoundation_{i} - U_{i}^{Found} \cdot (1-Foundation_{i,f})\\
    &FA_{i,f}\geq 0
\end{align}

Donde:
    \begin{itemize}
        \item $U_{i}^{foundation}= 0.6LotArea_{i}$

    \end{itemize}


    \item {Dormitorios}

    \item {Dormitorios 1er piso}
\begin{align}
    &AreaBedroom1_{i} \leq U_{max}^{Bed1} \cdot Floor1_{i}\\
    &AreaFullBath1_{i} \leq U_{max}^{FullB1} \cdot Floor1_{i}\\
    &AreaHalfBath1_{i} \leq U_{max}^{HalfB1} \cdot Floor1_{i}\\
    &AreaKitchen1_{i} \leq U_{max}^{Kitchen1} \cdot Floor1_{i}\\
    &AreaOther1_{i} \leq U_{max}^{Other1} \cdot Floor1_{i}\\
\end{align}

    \item {Dormitorios segundo piso}
\begin{align}
    &AreaBedroom2_{i} \leq U_{max}^{Bed2} \cdot Floor2_{i}\\
    &AreaFullBath2_{i} \leq U_{max}^{FullB2} \cdot Floor2_{i}\\
    &AreaHalfBath2_{i} \leq U_{max}^{HalfB2} \cdot Floor2_{i}\\
    &AreaKitchen2_{i} \leq U_{max}^{Kitchen2} \cdot Floor2_{i}\\
    &AreaOther2_{i} \leq U_{max}^{Other2} \cdot Floor2_{i}\\
\end{align}
\\
Donde:
    \begin{itemize}
        \item $U_{max}^{Bed2}=200 ft$
        \item $U_{max}^{FullB2}=60$
        \item $U_{max}^{HalfB2}=20$
        \item $U_{max}^{Kitchen2}=$
    \end{itemize}
\\
\begin{align}
    AreaBedroom2_{i}+AreaKitchen2_{i}+AreaHalfBath2_{i}+AreaFullBath2_{i}+AreaOther2_{i}\leq 2ndFlrSF_{i}
\end{align}
\\
\begin{align}
    Remainder2_{i}\geq 0\\
    AreaBedroom2_{i}+AreaKitchen2_{i}+AreaHalfBath2_{i}+AreaFullBath2_{i}+AreaOther2_{i}+Remainder2_{i}= 2ndFlrSF_{i}
\end{align}\\
Donde:
    \begin{itemize}
        \item $Remainder2_{i}$ Representa pasillos, closets
    \end{itemize}
\\

    \item{Area Living/Recreación}
\begin{align}
    &OtherRooms_{i}=OtherRooms1_{i}+OtherRooms2_{i}\\
    &AreaOther_{i}=AreaOther1_{i}+AreaOther2_{i}
\end{align}
Donde:\\
    \begin{itemize}
        \item $a_{min}^{other}=100$
        \item $R_{max}^{others}=8$
        \item $U_{i}^{others}=2ndFlrSF_{i}$
    \end{itemize}
    \item{Incluir estas areas en pisos}
\begin{align}
    TotalRmsAbvGrd_{i}=Bedroom_{i}+FullBath_{i}+HalfBath_{i}+OtherRooms_{i}\\
    OtherRooms_{i}=OtherRooms1_{i}+OtherRooms2_{i}\\
    AreaOther_{i}=AreaOther1_{i}+AreaOther2_{i}
\end{align}
    \item {Presencia mínima en primer piso}
\begin{align}
    OtherRooms1_{i}\geq 1
\end{align}
    \item {Activación en segundo piso}
\begin{align}
    OtherRooms2_{i}\leq R_{max}^{Other}\cdot Floor2_{i}\\
    AreaOther2_{i}\leq U_{i}^{Other2}
\end{align}
    \item {Mínimo de area}
\begin{align}
    AreaOther1_{i}\geq a_{min}^{other}\cdot OtherRooms1_{i}\\
    AreaOther2_{i} \geq a_{min}^{other}\cdot OtherRooms2_{i}
\end{align}


    \item {Perímetro casa para calcular Exterior}
\begin{align}
    &P_{i}^{(1)}\geq P_{i}^{(2)}
\end{align}
    \item Area Exterior a cubrir:\\
\begin{align}
    AreaExterior_{i}=H^{ext}\cdot (P_{i}^{(1)}+P_{i}^{(2)})
\end{align}
Donde:
    \begin{itemize}
        \item $H^{ext}$=7 ft
        \item $P_{i}^{(1)}\leq 2\cdot \Big ( \frac{1stFlrSF_{i}}{s_{min}}+s_{min}\Big)\cdot Floor1_{i}$
        \item $P_{i}^{(2)}\leq 2\cdot \Big ( \frac{2ndFlrSF_{i}}{s_{min}}+s_{min}\Big)\cdot Floor2_{i}$
        \item $P_{i}^{(1)}\geq 2\cdot \Big ( \frac{1stFlrSF_{i}}{s_{max}}+s_{max}\Big)\cdot Floor1_{i}$
        \item $P_{i}^{(2)}\geq2\cdot \Big ( \frac{2ndFlrSF_{i}}{s_{max}}+s_{max}\Big)\cdot Floor2_{i}$
        \item $s_{min}=20$
        \item $s_{max}=70$
    \end{itemize}
Este tipo de relación es estándar en la literatura de geometría arquitectónica y modelación espacial \cite{Durst2024}, donde el perímetro se expresa como una función del área y la proporción mínima de lados para mantener formas constructivamente viables \cite{Smith2017}. Y los $s_{min}$y $s_{max}$ fueron sacados de la base de datos \cite{decock2011}.\\
\\
Hago nueva variable W para linealizar:\\
\begin{align}
    &W_{i,e1}\leq AreaExterior_{i}\\
    &W_{i,e1}\leq U_{i}^{ext}\cdot Exterior1st_{i,e1}\\
    &W_{i,e1}\geq AreaExterior_{i}-U_{i}^{ext}(1-Exterior1st_{i,e1})\\
    &\sum_{e1}W_{i,e1}=AreaExterior_{i}
\end{align}\\
Donde:\\
    \begin{itemize}
        \item $U_{i}^{ext}=4500$
    \end{itemize}




    \item{Parámetros de superficie real del techo}

El parámetro $\gamma_{s,m}$ ajusta el área en planta cubierta por el techo
($PlanRoofArea_i$) para reflejar el área real de material requerido, considerando
la pendiente, el solape y la geometría asociada al estilo y material del techo.
Los valores se basan en estándares de
la \cite {ARMA2021} y el \cite{RoofingAlliance2022}.

\begin{table}[H]
\centering
\caption{Factores de superficie real del techo ($\gamma_{s,m}$) según estilo y material.}
\label{tab:roof_gamma}
\begin{tabular}{lllcc}
\toprule
\textbf{Estilo ($s$)} & \textbf{Material ($m$)} & \textbf{Descripción} & $\boldsymbol{\gamma_{s,m}}$ & \textbf{Fuente}\\
\midrule
Flat        & Membran      & Techo plano o de losa con mínima pendiente & 1.00 & NAHB (2023)\\
Flat        & CompShg      & Plano con tejas asfálticas & 1.05 & ARMA (2021)\\
Gable       & CompShg      & A dos aguas estándar (4:12–6:12 pitch) & 1.10 & NAHB (2023)\\
Gable       & Metal        & A dos aguas con panel metálico & 1.12 & Roofing Alliance (2022)\\
Hip         & CompShg      & A cuatro aguas (moderada pendiente) & 1.15 & NAHB (2023)\\
Hip         & Metal        & A cuatro aguas con panel metálico & 1.18 & Roofing Alliance (2022)\\
Gambrel     & WdShake      & Tipo granero, tejas de madera & 1.25 & NAHB (2023)\\
Mansard     & CompShg      & Mansarda con inclinación alta & 1.28 & NAHB (2023)\\
Shed        & Metal        & Techo inclinado de una sola vertiente & 1.12 & ARMA (2021)\\
Gable       & ClyTile      & A dos aguas con tejas de arcilla & 1.20 & Roofing Alliance (2022)\\
Hip         & TarGrv       & A cuatro aguas con grava asfáltica & 1.10 & NAHB (2023)\\
\bottomrule
\end{tabular}
\end{table}

Los valores típicos oscilan entre $1.00$ (techo plano) y $1.30$ (techo muy inclinado
o con múltiples vertientes). En la práctica, $\gamma_{s,m}$ puede estimarse como:
\[
\gamma_{s,m} \approx 1 + 0.1 \cdot \tan(\theta_s)
\]
donde $\theta_s$ es el ángulo medio de pendiente del estilo de techo.

\paragraph{Fuentes bibliográficas:}
\begin{itemize}
    \item National Association of Home Builders (NAHB). (2023). \textit{Residential Construction Guidelines, 2023 Edition.} Washington, D.C.
    \item Asphalt Roofing Manufacturers Association (ARMA). (2021). \textit{Residential Asphalt Roofing Manual, 2021 Edition.}
    \item Roofing Alliance. (2022). \textit{Technical Guide to Roof System Performance and Design.} National Roofing Contractors Association.
\end{itemize}

