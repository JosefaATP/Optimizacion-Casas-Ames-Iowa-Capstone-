Para algunas de las restricciones más relevantes que se presentan a continuación nos basamos en lo que mencionan \cite{Bertsimas2017}.\\
- $x_{i} \in [0, 1]^{p}$\\
- $z_{it}$: binaria que indica si $x_{i}$ esta en el nodo t.\\
- $l_t$: binaria que indica si la hoja t contiene algún punto. \\
- $N_{\min}$: número mínimo de puntos en cada hoja.\\
- $c_{kt}$: binaria que indica si la clase asignada al nodo t es k.\\
- $p_{t}:$ predicción del nodo t.\\
- $A_{L}(t):$ conjunto de ancestros de t cuya rama izquierda se ha seguido en el camino desde el nodo raíz hasta t.\\
- $A_{R}(t):$ conjunto de ancestros de la rama derecha.\\
- $T_{H}:$ nodos hoja.\\
- $T_{R}:$ nodos rama.\\
- $M_{1}, M_{2}$: número ssuficientemente grandes.\\
- $\epsilon > 0$
\begin{itemize}
    \item Número mínimo de puntos en cada hoja.\[
z_{it} \le l_t, \qquad \forall t \in T_R
\]
    \[
\sum_{i=1}^{n} z_{it} \ge N_{\min} \, l_t, \qquad \forall t \in T_R
\]
    \item Cada punto se asigna a una hoja.
    \[
\sum_{t \in T_H} z_{it} = 1, \quad i = 1, \dots, n
\]
    \item Divisiones estructurales para la estructura del árbol al asignar puntos a las hojas.\[
a_{m}^{T} x_i \ge b_{m} - M_{2}(1 - z_{it}), \quad i = 1,\ldots,n,\ \forall t \in T_R,\ \forall m \in A_R(t)
\]
    No se puede quedar como inecuación por eso se agrega un $\epsilon$ que es un párametro pequeño para este otro caso.
    \[
a_{m}^{T} x_i + \varepsilon \le b_{m} + M_{1}(1 - z_{it}), 
\quad i = 1,\ldots,n,\ \forall t \in T_R,\ \forall m \in A_L(t)
\]
    \item Solo una predicción de clase en cada nodo hoja que contenga puntos. \[
\sum_{k=1}^{K} c_{kt} = l_t, \qquad \forall t \in T_H
\]
    \item Predicción para cada i.
    \[
    \sum_{t \in T_{H}}p_{t} z_{it}, \quad  i = 1, \ldots,n\
    \]
\end{itemize}
