La tasación de viviendas es un proceso inherentemente complejo, influenciado por múltiples factores como la estructura, la ubicación y el entorno social. Los métodos tradicionales de valoración, a menudo basados en juicios subjetivos, generan discrepancias con el valor real de mercado, lo que provoca desconfianza e ineficiencia en el sector inmobiliario. A esta problemática se suma la dificultad que enfrentan los propietarios al decidir sobre el diseño y la remodelación bajo restricciones presupuestarias, buscando siempre optimizar la inversión para una mayor rentabilidad. Estas cuestiones impactan directamente en el funcionamiento del mercado y en la experiencia de compradores y vendedores.

La importancia de contar con una tasación precisa y una toma de decisiones adecuada en remodelación reside en la capacidad de permitir transacciones más justas e informadas, evitando la sobrevaloración y la subvaloración. Esto promueve un mercado inmobiliario más transparente y eficiente. Este desafío es particularmente relevante en Ames, Iowa, debido a su dinámica de alta demanda habitacional influenciada por la población joven y la presencia universitaria. En este contexto, el objetivo de este informe es diseñar un modelo de decisión para el mercado inmobiliario que logre estimar el valor justo de una vivienda con mayor exactitud y proponer soluciones óptimas de diseño y remodelación orientadas a alcanzar la casa óptima.
De esta manera, se busca corregir las limitaciones de las prácticas de tasación actuales y aportar un enfoque innovador para el diseño de viviendas eficientes y alineadas con las preferencias del mercado.

Para abordar este problema, se utiliza la base de datos \textit{Ames Housing Dataset}, la cual recopila más de 2.900 transacciones y 80 variables de viviendas en Ames realizadas entre 2006 y 2010. Inicialmente, se procedió con un análisis exploratorio mediante la matriz de correlación, para así identificar las variables más relevantes para el estudio. Finalmente, se construyó un modelo de regresión lineal multivariable con el fin de analizar el comportamiento y la significancia estadística de estas variables seleccionadas.

La integración del modelo predictivo \textit{XGBoost} con el optimizador \textit{Gurobi} permitió unir el aprendizaje automático con la toma de decisiones bajo restricciones reales. A través de \texttt{gurobi\_ml}, los árboles de decisión se incorporaron como restricciones lineales en un modelo de programación entera mixta, capaz de maximizar la utilidad esperada considerando costos y limitaciones constructivas, ofreciendo una aproximación más coherente con el comportamiento del mercado.

A partir de esto se definió el caso de estudio: una forma alternativa de predecir los precios mediante el modelo de predicción XGBoost. Para poder utilizar este modelo, fue necesario realizar una investigación sobre los hiperparámetros más relevantes, así como establecer un método para determinar los valores que mejor se ajustaran al problema sin incurrir en mayores riesgos que afectaran las predicciones. Este modelo, una vez entrenado, se implementaría en la siguiente fase de optimización en Gurobi.


Se desarrollan dos modelos de optimización con enfoques complementarios: el primero se orienta a la remodelación de una vivienda existente dentro de la base de datos, mientras que el segundo aborda la construcción de una vivienda desde cero. Ambos modelos se formulan considerando un conjunto de supuestos y limitaciones constructivas que delimitan el alcance de las decisiones y aseguran la coherencia técnica de los resultados.


Los resultados fueron coherentes con la teoría económica: la utilidad aumentó con el presupuesto, aunque con rendimientos decrecientes y una posterior saturación del modelo. Se identificaron proyectos “estrella” y barrios con mayor potencial de valorización, como \textit{NoRidge}, \textit{NridgHt} y \textit{StoneBr}. En conjunto, el modelo mostró un desempeño adecuado para apoyar decisiones de remodelación y estimar con mayor precisión el valor de las viviendas.
