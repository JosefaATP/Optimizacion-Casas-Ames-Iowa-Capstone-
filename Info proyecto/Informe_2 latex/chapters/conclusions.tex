\section{Pasos futuros \label{sec:sec1}}
Los próximos pasos se centrarán en fortalecer la fiabilidad predictiva de $\mathbf{XGBoost}$ implementando una validacion más robusta para los rangos de hiperparametros, implementar opciones en la calibracion para el early stoping, y explorar si es pertinente otras alternativas a la optimizacion bayesiana para la obtencion de hiperparametros, asi como asegurar la correcta implementación y el análisis crítico de la optimización con $\mathbf{Gurobi}$. En la fase predictiva, la prioridad será mitigar el $\mathbf{\text{riesgo de sobreajuste}}$ inherente a la arquitectura del modelo, lo que se abordará mediante la implementación de una $\mathbf{\text{validación cruzada robusta}}$. Adicionalmente, para validar la capacidad de $\mathbf{\text{generalización}}$ del modelo fuera de los datos de entrenamiento, se realizarán pruebas de predicción utilizando un conjunto de $\mathbf{\text{datos externo}}$ (por ejemplo, de un portal inmobiliario). En el análisis de desempeño, se completará la evaluación de $\mathbf{KPIs}$ mostrando $\mathbf{\text{curvas de convergencia}}$ para todas las métricas ($\mathbf{R^2}$, $\mathbf{\text{RMSE}}$, $\mathbf{\text{MAE}}$) y se graficará la $\mathbf{\text{distribución de errores}}$ para analizar el $\mathbf{\text{sesgo}}$ del modelo.

Respecto a la fase de optimización, resulta crítico configurar correctamente el parámetro $epsilon$ de Gurobi, investigando su valor óptimo para evitar que el \textit{solver} ignore las particiones de los árboles de decisión y genere resultados incorrectos. Además, se procederá a $\mathbf{\text{formalizar la integración del Modelo de Construcción en el \textit{framework} XGBoost-Gurobi}}$, asegurando que este también utilice la predicción del valor de la vivienda para optimizar la rentabilidad en la creación de una casa óptima desde cero. Será requisito formal incluir la $\mathbf{\text{formulación matemática completa}}$ que Gurobi utiliza para representar los árboles de $\mathbf{XGBoost}$ como un sistema de desigualdades. Finalmente, para eliminar el $\mathbf{\text{sesgo de escala en el análisis de \textit{ROI}}}$, se $\mathbf{\text{corregirá la fórmula de dicho indicador}}$ para expresarla como un porcentaje y se migrará de un presupuesto fijo a un $\mathbf{\text{presupuesto dinámico}}$ (ej., un porcentaje del valor inicial de la propiedad). 
\\Finalmente, como parte de la estrategia para garantizar la robustez y la validez práctica de las recomendaciones, se planea $\mathbf{\text{escalar la aplicación de los modelos de optimización}}$. El análisis computacional inicial se realizó sobre una muestra de $\mathbf{30 \ viviendas}$ para obtener resultados preliminares y validar la formulación. Sin embargo, los próximos pasos contemplan $\mathbf{\text{extender este análisis a un conjunto de } 120 \ \text{casas}}$ representativas del \textit{Ames Housing Dataset}. Este escalamiento se aplicará tanto al $\mathbf{\text{Modelo de Renovación}}$ como al $\mathbf{\text{Modelo de Construcción}}$, permitiendo así generar un conjunto de $\mathbf{\text{soluciones óptimas significativamente más amplio}}$ y estadísticamente más robusto. 

\section{Conclusión \label{sec:sec2}}
El presente proyecto demuestra la $\mathbf{\text{viabilidad y la superioridad analítica}}$ de integrar modelos de \textit{Machine Learning} avanzados con técnicas de Programación Matemática para resolver problemas estructurales de ineficiencia y desinformación en el mercado inmobiliario. La inexactitud en la tasación y la complejidad en la toma de decisiones de diseño y remodelación, magnificada en escenarios dinámicos como $\mathbf{\text{Ames, Iowa}}$, exigen la adopción de enfoques que superen las limitaciones de los métodos tradicionales. La $\mathbf{\text{rigurosa etapa de depuración y análisis}}$ del $\textit{Ames Housing Dataset}$, que incluyó la $\mathbf{\text{limpieza y el ajuste por inflación a valores de 2025}}$, sentó una base de datos coherente y libre de inconsistencias, esencial para un modelamiento confiable.

La $\mathbf{\text{metodología híbrida}}$, basada en la sinergia de $\mathbf{XGBoost}$ y $\mathbf{Gurobi}$, se posiciona como el núcleo de la solución. $\mathbf{XGBoost}$, elegido por su capacidad de $\mathbf{\text{capturar relaciones no lineales}}$, superó el $\mathbf{\text{sesgo y el error}}$ de los modelos lineales iniciales, ofreciendo una $\mathbf{\text{estimación de precio precisa}}$. Esta predicción fue el pilar para la optimización realizada por $\mathbf{Gurobi}$, que, a través de la $\mathbf{\text{Programación Entera Mixta (MIP)}}$, no solo predijo, sino que $\mathbf{\text{prescribió la decisión óptima}}$ en los Modelos de Remodelación y Construcción. Los resultados preliminares demuestran una $\mathbf{\text{alta eficiencia en capital}}$ y un $\mathbf{\text{incremento promedio del } 39\% \ \text{en el valor de la vivienda}}$, lo que valida el objetivo de guiar a los vendedores hacia $\mathbf{\text{inversiones estratégicas y rentables}}$.

Finalmente, la consolidación del proyecto se centra en abordar las $\mathbf{\text{limitaciones identificadas}}$ para garantizar la $\mathbf{\text{máxima robustez y validez}}$ de la herramienta. Esto incluye mitigar el riesgo de sobreajuste de $\mathbf{XGBoost}$ con validación externa, $\mathbf{\text{eliminar el sesgo de escala}}$ mediante un $\mathbf{\text{presupuesto dinámico}}$ y $\mathbf{\text{escalar la optimización de } 30 \ \text{a } 120 \ \text{casas}}$. Avanzar hacia herramientas de predicción y optimización tan robustas no es solo una necesidad técnica, sino una $\mathbf{\text{condición indispensable}}$ para fomentar un mercado $\mathbf{\text{más transparente, equitativo y alineado}}$ con las necesidades reales de los actores, asegurando así el desarrollo $\mathbf{\text{sostenible y la modernización}}$ del sector inmobiliario.
