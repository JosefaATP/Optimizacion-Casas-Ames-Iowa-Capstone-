El mercado inmobiliario es caracterizado por su complejidad y dinamismo, donde el darle un determinado valor a una vivienda representa un desafío central tanto para compradores como vendedores. Los métodos tradicionales de tasación, sustentados principalmente en referencias de ventas anteriores y criterios subjetivos, suelen generar gran discrepancia respecto al valor real del inmueble, lo que se traduce en ineficiencias, desconfianza y grandes pérdidas económicas para los agentes involucrados \cite{evans2019}. Este problema constituye un dolor estructural del sector, ya que afecta la transparencia del mercado y limita la capacidad de tomar decisiones informadas, en consecuencia, la tarea de obtener una tasación objetiva de una vivienda se torna en un desafío particularmente complejo de resolver.

En esta misma línea, el mercado inmobiliario también impone una amplia responsabilidad en la toma de decisiones por parte de los propietarios. Con frecuencia, quienes buscan vender su vivienda realizan remodelaciones con el propósito de incrementar su rentabilidad. No obstante, dichas intervenciones no siempre cumplen las expectativas planteadas. Según \citeA{dunaway2024}, un 30\% de las renovaciones efectuadas en Estados Unidos tienen como objetivo aumentar el valor de la propiedad; sin embargo, un 24\% de los propietarios se arrepiente por el elevado gasto que implicaron y un 16\% por haber incurrido en deudas. Estos datos reflejan que los propietarios suelen invertir en remodelaciones sin contar con claridad respecto al impacto real de dichas mejoras en el valor de la vivienda, lo que conlleva riesgos financieros y genera frustración.

Un caso de estudio interesante a analizar es el de la ciudad de Ames, ubicada en el estado de Iowa, Estados Unidos. En Ames la valoración óptima de una vivienda y la toma de decisiones en diseño y remodelación representa un desafío considerable debido a la influencia de factores políticos, económicos y sociales. Según \citeA{ye2024}, dichos fenómenos son causados por la marcada presencia de la universidad Iowa State University, que convierte a esta ciudad en un centro de atracción para estudiantes y profesionales. Esto genera una dinámica de alta demanda de inmuebles en la zona, lo que acentúa la necesidad de tasaciones precisas y decisiones óptimas de diseño y remodelación de una vivienda, lo que resalta la importancia de contar con información adecuada sobre las preferencias de los compradores a fin de satisfacer sus requerimientos habitacionales.

Considerando el desafío que implica lograr una tasación óptima de viviendas en Ames, junto con la necesidad de tomar decisiones acertadas en materia de diseño y remodelación, se plantea el desarrollo de un proyecto orientado a construir un modelo robusto de soporte para la toma de decisiones en el ámbito inmobiliario. Cuyo objetivo principal es predecir con precisión el valor de una vivienda y, adicionalmente, recomendar remodelaciones óptimas bajo restricciones presupuestarias. Con ello, se busca establecer un precio justo que contemple las necesidades y disposición de pago de los compradores, maximizando al mismo tiempo el valor y la rentabilidad de la propiedad. 

El objetivo central es desarrollar una herramienta integral que, apoyada en el poder predictivo de XGBoost y la optimización de Gurobi, permita a los actores del mercado inmobiliario tomar decisiones estratégicas. Esta herramienta buscará no solo proveer a los compradores con la vivienda óptima que satisfaga sus necesidades a un precio justo, sino también brindar a los vendedores una guía precisa de diseño y remodelación para maximizar la rentabilidad de sus propiedades. Para lograr esto, se empleará XGBoost para la estimación precisa del valor de la vivienda, el cual servirá como base para dos modelos de optimización distintos, implementados con Gurobi: el Modelo de Remodelación, que se centra en una casa ya existente y determina cuáles son las remodelaciones óptimas que incrementarán su rentabilidad; y el Modelo de Construcción, diseñado para definir la casa óptima desde cero, especificando la combinación de características de diseño que generan la mayor rentabilidad esperada. Gurobi trabajará iterativamente con el predictor XGBoost, explorando y evaluando un vasto espacio de combinaciones de características de vivienda para finalmente proponer los cambios y diseños que aseguren la máxima rentabilidad bajo las restricciones de presupuesto o mercado.


Para evaluar de manera rigurosa el cumplimiento de los objetivos del proyecto, es fundamental definir un conjunto de indicadores clave de desempeño (KPIs) que permitan medir tanto la precisión del modelo de tasación como la eficiencia económica de las soluciones de optimización propuestas. En el caso de los modelos predictivos, la evaluación se realizará mediante métricas estadísticas ampliamente aceptadas. El coeficiente de determinación ($R^2$) medirá la proporción de variabilidad del precio explicada por el modelo, mientras que el error cuadrático medio (RMSE) permitirá detectar desviaciones de gran magnitud con impacto financiero. Complementariamente, el error absoluto medio (MAE) y el error porcentual absoluto medio (MAPE) cuantificarán la magnitud promedio de los errores en términos absolutos y relativos, respectivamente, siendo este último especialmente útil para comparar la precisión entre propiedades de distintos valores.

En segundo lugar, se definen dos indicadores clave de desempeño (KPIs) para evaluar la optimalidad del proyecto. Con el objetivo de analizar el impacto económico de las intervenciones, se propone medir el porcentaje neto de mejora del valor del inmueble, calculado como la diferencia entre el valor final y el valor inicial de la vivienda, dividida por su valor inicial. Un bajo cumplimiento de este KPI indicaría que las inversiones realizadas no generan un incremento significativo en el valor de la propiedad, lo que sugiere una intervención poco eficiente o con una selección inadecuada de mejoras.

Finalmente, para el objetivo económico de asegurar la rentabilidad de las remodelaciones y la optimización de los recursos, se utiliza el $\mathbf{\text{ROI}}$ ($\mathbf{\text{Return on Investment}}$), considerado en relación con el $\mathbf{\text{Porcentaje de Presupuesto Usado}}$. Este par de indicadores es el más pertinente para medir la eficiencia: el ROI es la métrica estándar para determinar la relación entre el beneficio (incremento del valor de la propiedad) y la inversión (costo de la remodelación). Al compararlo con el presupuesto utilizado, se evalúa si el modelo Gurobi está logrando soluciones que no solo maximizan la ganancia neta, sino que lo hacen de la manera $\mathbf{\text{más eficiente en capital}}$, priorizando aquellas remodelaciones que aportan mayor valor adicional por cada unidad monetaria invertida, minimizando así el riesgo de sobreinversión.\\


